\chapter{実装}
\label{chap:implementation}

本章では、第\ref{chap:design}章で述べたプログラミング環境の構成要素について述べる。

\section{Babascript}

\subsection{基本仕様}
人オブジェクトを宣言する

人オブジェクトに対するメッセージ送信によって、プログラム上にて人とコミュニケーションを取る

\subsection{人への命令構文}
人への命令構文は、人オブジェクトに対して直接メソッド実行をすることによってできる。

プログラミング言語Rubyなどにおけるmethodmissingと同じ機能を利用する

メソッド名とその引数が命令の内容として送信される

\subsection{オプション情報の付加}

メソッド名以外にも情報として渡したいときには、第一引数にオプション情報を付与する

フォーマット情報であったり、その他様々な情報を付与する

\subsection{コールバック関数}

命令構文の第二引数にコールバック関数を指定すると、実行結果を取得した後にこのコールバック関数が呼ばれる

人間は計算機の処理に比べて遅延しがちであるため、非同期を前提とした実装をしている

\subsection{タスク情報}

以下のような情報でタスクオブジェクトは構成される。

\section{Babascript Client}

\section{通信方式}

\section{プラグイン機構}
