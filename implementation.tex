\chapter{実装}
\label{chap:implementation}

本章では、第\ref{chap:design}章で述べたプログラミング環境の構成要素について述べる。

\section{Babascript}

\subsection{基本仕様}
人オブジェクトを宣言する

人オブジェクトに対するメッセージ送信によって、プログラム上にて人とコミュニケーションを取る

\subsection{人への命令構文}
人への命令構文は、人オブジェクトに対して直接メソッド実行をすることによってできる。

プログラミング言語Rubyなどにおけるmethodmissingと同じ機能を利用する

メソッド名とその引数が命令の内容として送信される

\subsection{オプション情報の付加}

メソッド名以外にも情報として渡したいときには、第一引数にオプション情報を付与する

フォーマット情報であったり、その他様々な情報を付与する

\subsection{コールバック関数}

命令構文の第二引数にコールバック関数を指定すると、実行結果を取得した後にこのコールバック関数が呼ばれる

人間は計算機の処理に比べて遅延しがちであるため、非同期を前提とした実装をしている

\subsection{タスク情報}

以下のような情報でタスクオブジェクトは構成される。

\section{Babascript Client}

Babascript Agent Applicationは、プログラムと人とのインタラクションを仲介する役割を持つ。

機能としては、プログラムからの命令を受け取ることと、命令に対する人の処理結果を返すこと

Babascript Clientは、サービス部とインタフェース部から構成される。

\subsection{サービス}

命令の受け取りや返り値の送信などを担う

\subsection{インタフェース}

ユーザとのインタラクションを行う。
命令をユーザに見せるのと、実際に実行結果を入力させる機能を持つ

\subsubsection{Webアプリケーション}
\subsubsection{Android Wear アプリケーション}

\section{通信方式}

Babascript及びBabascript Clientは、通信手法を切り替えることが出来る
この通信モジュール部分をBabascript Adapterと呼ぶ。

% ここの区分、どうしよ?
% Node-lidaの利用は前提とする?
% Node-linda Socket.IO Adapter
% Node-linda Pushnotification Adapter
% こんな感じの分け方でいい気がする。
% 今はnode-lindaを利用してるけど、Adapterは非常に簡単に開発できる、みたいな。

\subsection{Node-Linda Adapter}

Node-Linda Adapterは、Socket.IOを用いてNode-Lindaというプラットフォームに接続し、

\subsection{PushNotification Adapter}

Node-Linda Pushnotification Adapter は、HTTP RequestとPushnotificationを用いて
Node-Linda プラットフォームに接続する。

\section{プラグイン機構}

Babascript 及びBabascriptClientはその機能を拡張するために、プラグイン機構を持つ。

以下の様に使うことで、Babascript及びBabascriptClientによってイベントが発生した時に、
それに応じたデータを受け取り、自由に操作することができる。

Babascript及びBabascriptClientは、以下のイベントを受け取る。

- initialize
- connect
- send
- receive

\subsection{具体例}

\subsubsection{Logger Plugin}

\subsubsection{Datasync Plugin}
