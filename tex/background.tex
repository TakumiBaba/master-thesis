\chapter{研究背景}\label{chap:background}

本章では、本研究の背景について述べる。

\section{プログラムとは}\label{ux30d7ux30edux30b0ux30e9ux30e0ux3068ux306f}

\subsection{コンピュータプログラム}\label{ux30b3ux30f3ux30d4ux30e5ux30fcux30bfux30d7ux30edux30b0ux30e9ux30e0}

コンピュータに実行させたい処理を記述するための手順書として、コンピュータプログラムが存在する。

コンピュータにとってのプログラムとは、コンピュータが実行すべき処理の内容の手順を示している
手順書である。 プログラマは

\subsection{人にとってのプログラム}\label{ux4ebaux306bux3068ux3063ux3066ux306eux30d7ux30edux30b0ux30e9ux30e0}

人にとってのプログラムとしては、マニュアルやレシピといったものが存在する。

\subsection{両者の類似性}\label{ux4e21ux8005ux306eux985eux4f3cux6027}

人にとってのプログラムであるマニュアルやレシピといったものは、分解してあげた上で
コンピュータプログラムのように記述してみると、非常に類似している。
両者を同じようなフォーマットの上で書いてみても、何も違和感ないと思う。

\section{プログラムの処理領域の拡大}\label{ux30d7ux30edux30b0ux30e9ux30e0ux306eux51e6ux7406ux9818ux57dfux306eux62e1ux5927}

\subsection{コンピュータ上から実世界にも}\label{ux30b3ux30f3ux30d4ux30e5ux30fcux30bfux4e0aux304bux3089ux5b9fux4e16ux754cux306bux3082}

\begin{itemize}
\itemsep1pt\parskip0pt\parsep0pt
\item
  センサ・アクチュエータの制御など
\item
  実世界もプログラムで制御する
\end{itemize}

\subsection{計算資源としての人}\label{ux8a08ux7b97ux8cc7ux6e90ux3068ux3057ux3066ux306eux4eba}

\begin{itemize}
\itemsep1pt\parskip0pt\parsep0pt
\item
  人にコンピュータの代わり計算させる
\item
  人はプログラムにとって、処理を行うリソースなのでは
\end{itemize}

\section{人と計算機の融合的プログラミング環境}\label{ux4ebaux3068ux8a08ux7b97ux6a5fux306eux878dux5408ux7684ux30d7ux30edux30b0ux30e9ux30dfux30f3ux30b0ux74b0ux5883}

もはや人間はプログラムの要素として普通なのでは
実世界を操作するのに、人間を使わない手はない
実世界を操作するのに、人間と計算機、双方のリソースを使う
コンピュータが得意なことはコンピュータが、人が得意なことは人がやれば良い
これによって、今まではプログラムとして記述してこなかったような、人間の生活さえもプログラミングできるのでは
また、可能な限りコンピュータに処理を実行させて、人間は人間にしかできないことに専念できるのでは?

問題は、プログラムからある特定の人物に対する処理命令を送るのがめんどくさいこと
クラウドソーシングとかで人間への処理命令を送ることはできるけど、クラウドソーシングプラットフォームでは
家族や自分を指定して何かやらせるといったことは困難。
特殊な記法なく、今までどおりの普通にプログラミングしてるだけなのに、日常生活をプログラミングしたい

\section{まとめ}\label{ux307eux3068ux3081}

本章では、人間とコンピュータが

--\textgreater{}
