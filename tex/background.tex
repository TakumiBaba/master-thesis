\chapter{背景}
\label{chap:background}

本章では、

\section{コンピュータプログラムと人のプログラム}\label{ux30b3ux30f3ux30d4ux30e5ux30fcux30bfux30d7ux30edux30b0ux30e9ux30e0ux3068ux4ebaux306eux30d7ux30edux30b0ux30e9ux30e0}

\subsection{計算資源としての人}

コンピュータは非常に優秀な計算機器であるが、コンピュータだけでは解決が困難な問題は現在の技術では存在する。
このコンピュータだけでは処理が困難な問題に関して、人間に処理させることによって解決しようという考えは
ヒューマンコンピュテーション\cite{humancomputation}と呼ばれ、近年注目を集めている。
つまり、人間をコンピュータと同じ計算資源として考え、処理させるということである。

\cite{recaptcha}

HumanComputation /
Crowdsourcingなどによって、人は明確に計算資源として扱われるようになった。
今後も、コンピュータにできないことは人にやらせることで問題解決する手法は使われ続けると考えられる。

\subsection{コンピュータプログラムと人のプログラムの類似性}

\subsection{}

\subsection{対象を明示的に指定できない}

\subsection{シンプルでない}

\% - ヒューマンコンピュテーションの登場 \% -
人間も計算資源として利用することが提案されている。 \% \% -
HumanComputationの登場 \% - 人間は明確に計算資源となる \% -
人と計算機を同じように扱う \% -
人と計算機、双方に対する処理命令フォーマットが異なる \% -
計算機はプログラムの通りに動く \% -
人はマニュアル等に沿って動くことが多い \% -
実行可能なフォーマットに統一するべきでは \% - プログラムで記述できる \%
- マニュアル等は非常にプログラム的 \% - 例えば、運動会プログラムとか \%
- 社会の多くはプログラムによって支配されている \% -
プログラムから人を扱う \% - 様々な研究 \% -
クラウドソーシング系のばっかり \% -
演算のための機能としてしか利用されることはない \% -
人は汎用的実行主体である \% -
演算だけでなく、様々なことができるし、している \% -
マニュアルは様々なことが記述されている \% -
より需要があるのは、自分自身や家族、会社などの組織内の人間 \% -
身近な人間をプログラムできるほうが良い \% -
普通にプログラムを書いてるかのように扱えるべき \% -
特定の身近な人をプログラムに組み込めるような仕組みはない \% -
人とコンピュータを同じフォーマットで扱う \% -
今までは人がやっていたけどコンピュータでもできそうなことを全てコンピュータにやらせる
\% - 以下の2つが考えられる \% - 人がプログラムに積極的に貢献する \% -
計算機だけでは難しい処理も実現する \% - 出来るだけ人のやることを減らす
\% - 可能な限りの処理を計算機に実行させる \% -
人には、本当に人がやるべきようなことをやらせる \% -
新しい技術によって人がやらなくても良いことが増えても、すぐに対応可能である
