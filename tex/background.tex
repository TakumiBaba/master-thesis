\chapter{研究背景}\label{chap:background}

本章では、本研究の背景について述べる。

\section{プログラムとは}\label{ux30d7ux30edux30b0ux30e9ux30e0ux3068ux306f}

プログラムとは。

\begin{itemize}
\itemsep1pt\parskip0pt\parsep0pt
\item
  プログラムは処理手順が記述されたもの
\item
  コンピュータプログラムは、コンピュータに実行してもらいたい処理を記述しておく手順書である。
\item
  一方で、人間にとってもプログラムというものは存在する。
\item
  例えば、マニュアルやレシピといったものだ。
\item
  また、音楽の演奏会のプログラムなども、演奏会全体の一連の処理の流れを記述しておくものである。
\item
  両者を比べると、非常に類似している
\item
  画像で示す
\item
  プログラムとは、実行したい処理を記述するものであり、その実行対象は選ばれない。
\item
  コンピュータのほうが得意だからコンピュータにやらせているだけであって、人間がやっても良いのである。
\item
  実際に、HumanComputationという概念もあり、プログラムにとっては結果さえ得られれば実行者はなんでも良くなってる
\end{itemize}

\section{プログラムの処理領域}\label{ux30d7ux30edux30b0ux30e9ux30e0ux306eux51e6ux7406ux9818ux57df}

\subsection{画面上から実世界へ}\label{ux753bux9762ux4e0aux304bux3089ux5b9fux4e16ux754cux3078}

プログラムによって制御を記述できるような領域はどんどん広がっている。
近年では、ArduinoやPhidgets、RaspberryPi等の登場によって、誰でも非常に簡単にセンサーやアクチュエータを扱えるようになっている。
その制御を記述するのはプログラムである。
また、建築物の構成要素をプログラマブルにする試みや、プログラムによってその構成を動的に変化させるモジュールについての研究もなされている。
今後、プログラムによる制御は更に広がり、あらゆることをプログラムで記述するようになっていくのではないかと考えられる。

\subsection{計算資源としての人}\label{ux8a08ux7b97ux8cc7ux6e90ux3068ux3057ux3066ux306eux4eba}

また、コンピュータのみでは処理できないような処理を実現するために人間を利用することで、プログラムが処理できる領域を拡大させてもいる。

\begin{itemize}
\itemsep1pt\parskip0pt\parsep0pt
\item
  人にコンピュータの代わり計算させる
\item
  人はプログラムにとって、処理を行うリソースなのでは
\end{itemize}

\section{コンピュータと人のプログラムに基づいた共同処理}\label{ux30b3ux30f3ux30d4ux30e5ux30fcux30bfux3068ux4ebaux306eux30d7ux30edux30b0ux30e9ux30e0ux306bux57faux3065ux3044ux305fux5171ux540cux51e6ux7406}

\begin{itemize}
\itemsep1pt\parskip0pt\parsep0pt
\item
  もはやコンピュータも人も、処理を実行する対象として同じように見るべき
\item
  処理に応じて、コンピュータが得意ならコンピュータがやれば良い
\item
  人が得意なら人がやれば良い
\item
  コンピュータと人がお互いの得意分野において力を発揮し、処理を実行していくようなプログラムが書ければ良い
\item
  だが、現状では、人はプログラムから利用されているとしても、知的労働のみである
\end{itemize}

\section{人的資源の利用が限られている}\label{ux4ebaux7684ux8cc7ux6e90ux306eux5229ux7528ux304cux9650ux3089ux308cux3066ux3044ux308b}

\begin{itemize}
\itemsep1pt\parskip0pt\parsep0pt
\item
  プログラムは汎用処理記述フォーマットである
\item
  プログラムはコンピュータだけでなく、人間への指示も記述可能である
\item
  両者の境界はなくなりつつある
\item
  人的資源を使って汎用的に処理を記述すれば良い
\item
  だけど、現状は人的資源は知的労働にのみしか使われていない
\item
  人間は知的労働のみならず、様々な処理が可能な存在である。
\item
  なぜそういったことができないか、人的資源を活用するプログラムはクラウドソーシングしかないから
\item
  クラウドソーシングでは、例えば自分自身を対象として処理をさせるとかができない
\item
  自分の身の周りの出来事を記述することができない
\item
  例えばレシピとか、そういったものをプログラムとして記述できない。
\end{itemize}

\section{人と計算機の融合的プログラミング環境}\label{ux4ebaux3068ux8a08ux7b97ux6a5fux306eux878dux5408ux7684ux30d7ux30edux30b0ux30e9ux30dfux30f3ux30b0ux74b0ux5883}

今後、人間はプログラムから呼び出されることが普通なようになっていくのではないか。
また、プログラムから制御しないような空間は今後どんどん減っていくのではないか。

もはや人間はプログラムの要素として普通なのでは
実世界を操作するのに、人間を使わない手はない
実世界を操作するのに、人間と計算機、双方のリソースを使う
コンピュータが得意なことはコンピュータが、人が得意なことは人がやれば良い
これによって、今まではプログラムとして記述してこなかったような、人間の生活さえもプログラミングできるのでは
また、可能な限りコンピュータに処理を実行させて、人間は人間にしかできないことに専念できるのでは?

問題は、プログラムからある特定の人物に対する処理命令を送るのがめんどくさいこと
クラウドソーシングとかで人間への処理命令を送ることはできるけど、クラウドソーシングプラットフォームでは
家族や自分を指定して何かやらせるといったことは困難。
特殊な記法なく、今までどおりの普通にプログラミングしてるだけなのに、日常生活をプログラミングしたい

\section{まとめ}\label{ux307eux3068ux3081}

本章では、人間とコンピュータが

--\textgreater{}
