\chapter{研究背景}
\label{chap:background}

本章では、

\section{コンピュータプログラムと人のプログラム}\label{ux30b3ux30f3ux30d4ux30e5ux30fcux30bfux30d7ux30edux30b0ux30e9ux30e0ux3068ux4ebaux306eux30d7ux30edux30b0ux30e9ux30e0}

\begin{itemize}
\item
  コンピュータプログラムとは
\item
  人にとってのプログラム
\item
\end{itemize}

\section{計算資源としての人}\label{ux8a08ux7b97ux8cc7ux6e90ux3068ux3057ux3066ux306eux4eba}

コンピュータだけでは処理が難しい問題も存在する。
それを人を計算資源として利用することで解決しようという考えかたはヒューマンコンピュテーション\cite{humancomputation}と呼ばれる
つまり、人間をコンピュータと同じ計算資源として捉え、各種システムに組み込むということである。
ヒューマンコンピュテーションは近年注目を集めており、多くの研究が行なわれている。

ヒューマンコンピュテーションを始めとし、計算機のみでは処理が困難な問題を人間を利用することで解決

\subsection{Human Computation}\label{human-computation}

\subsection{Human as Sensor}\label{human-as-sensor}

\subsection{Human as Actuator}\label{human-as-actuator}

\subsection{プログラミング環境}\label{ux30d7ux30edux30b0ux30e9ux30dfux30f3ux30b0ux74b0ux5883}

\section{既存手法の問題点}\label{ux65e2ux5b58ux624bux6cd5ux306eux554fux984cux70b9}

\begin{itemize}
\itemsep1pt\parskip0pt\parsep0pt
\item
  人を明示的に指定できない
\item
  シンプルかつ汎用的に利用できない
\end{itemize}

\section{人間と計算機の融合的プログラミング}\label{ux4ebaux9593ux3068ux8a08ux7b97ux6a5fux306eux878dux5408ux7684ux30d7ux30edux30b0ux30e9ux30dfux30f3ux30b0}

コンピュータプログラムと人のプログラムは、

\begin{itemize}
\itemsep1pt\parskip0pt\parsep0pt
\item
  両者の融合の可能性
\end{itemize}

\subsection{計算資源としての人}

コンピュータは非常に優秀な計算機器であるが、コンピュータだけでは解決が困難な問題は現在の技術では存在する。
このコンピュータだけでは処理が困難な問題に関して、人間に処理させることによって解決しようという考えは
ヒューマンコンピュテーション\cite{humancomputation}と呼ばれ、近年注目を集めている。
つまり、人間をコンピュータと同じ計算資源として考え、処理させるということである。

\cite{recaptcha}

HumanComputation /
Crowdsourcingなどによって、人は明確に計算資源として扱われるようになった。
今後も、コンピュータにできないことは人にやらせることで問題解決する手法は使われ続けると考えられる。

\subsection{コンピュータプログラムと人のプログラムの類似性}

\subsection{}

\subsection{対象を明示的に指定できない}

\subsection{シンプルでない}
