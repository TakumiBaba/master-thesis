\chapter{背景}\label{chap:background}

本章では、プログラムを実行する対象の変化や、その処理領域の変化について述べた後、
これらの変化がもたらす新しいプログラミングの領域について示す。

\section{計算資源としての人間}\label{sec:human-as-computational-resources}

コンピュータのみでは解決が困難な問題を、人間を計算資源として利用することによって解決する考え方は
ヒューマンコンピュテーション\cite{humancomputation}と呼ばれる。
コンピュータは優れた処理能力を有するが、パターン認識能力など人間のほうが得意な処理分野は多く存在する。
これらの分野において人間とコンピュータがお互いの得意な領域で力を発揮することによって、
今までは実現が困難であったような処理でも実現可能となった。

ヒューマンコンピュテーションの例として挙げられるシステムに、reCAPTCHA\cite{recaptcha}がある。
reCAPTCHAは、人間かコンピュータを識別するためのテストとして活用されているCAPTCHA\cite{captcha}を応用したものだ。
人間かコンピュータかの識別を行いながら、コンピュータでは識別が困難な文字の認識を人間に実行させるシステムだ。
reCAPTCHAを使うことで、利用者は自分が人間であることを証明しながら、紙の本のデジタル化に貢献している。

他の例として、Soylent\cite{soylent}といったソフトウェアも存在する。
Soylentは、文章の校正作業をインターネットを介して他の人間たちに実行させるシステムだ。
文章の校正も人間のほうが得意であったり、内容によっては人間にしかできない作業である。
この他にも、多くのヒューマンコンピュテーションの考えを活用したシステムが存在する。

インターネットを介した不特定多数の人間に仕事を依頼する仕組みはクラウドソーシング\cite{riseofcrowdsourcing}と呼ばれる。
必要なときに必要なだけの人材を安価に集めることができ、近年注目を浴びている。
クラウドソーシングにおいてもヒューマンコンピュテーションの考えが反映された事例は多く存在し、
大量の人間を計算資源とした処理が実現されている。

近年ではスマートフォンやタブレットデバイスの普及によって、人間はインターネットを介していつでもあらゆる場所で、
様々なシステムに接続可能となっている。
これは、利用者側にとっては便利にインターネットを介して情報を得られるようになったということであり、
システム側にとってはいつでも利用者とコミュニケーションが取れるようになったということでもある。
また、これらのデバイスは気軽に持ち運ぶことができるため、家の中や外出先等、どこでもやりとりができるようになっている。
つまり、いつでもどこでも人間を計算資源として利用する下地が出来つつあるということだ。
計算資源として常時アクセス可能になりつつある人間は、今まで以上に計算資源として利用されると考えられる。

ヒューマンコンピュテーションやクラウドソーシングの流行によって、
人間を優秀な計算資源として捉え活用する事例は増えている。
人間と計算機、どちらが実行するとしても、実現したい処理が可能で結果を得ることが出来れば問題はない。
ある処理を実現する上では、人間も計算機も処理を実行するための同じ資源であり、
計算機が得意なことは計算機が、人間が得意なことは人間が実行すれば良いということである。
人間と計算機は計算資源として等価になっていき、その境界が曖昧になっていくことが予想される。

\section{プログラムの制御領域}\label{sec:are-of-program}

世界中にある様々なコンピュータやデバイスがインターネットに繋がり、プログラムによって制御されるようになってきている。
従来ではプログラミングといえば、その対象は計算機の中や画面の中の世界の制御のみにとどまりがちだったが、
近年では、Arduino\footnote{http://www.arduino.cc/}やRaspberryPi\footnote{http://www.raspberrypi.org/}等の登場によって、
誰でも簡単にセンサーやアクチュエータを扱えるようになっている。
実世界から情報を得たり、実世界を制御するためのプログラムは今では誰でも簡単に扱うことができる。
プログラムは実世界を含んだ広い領域の制御のために使われていくと考えられる。

マーク・ワイザーが提唱したユビキタスコンピューティング\cite{weiser1991computer}は、実世界環境にコンピュータを溶けこませ、
ユーザは意識することなくコンピュータによる支援を享受できるという概念だ。
ユビキタスコンピューティングのように、日常生活をコンピュータを用いて支援する仕組みについて研究は多くなされているが、
それらの仕組みの多くはプログラムによって制御される。
類似の概念としてはInternet of
Things(以下、IoT)\cite{iot}といった考えも提唱されている。
あらゆるモノがインターネットに繋がり、情報をやりとりすることによって様々な恩恵を得ることができるというのが
IoTの基本的な考え方である。
IoTの考えに基づいて様々なモノが繋がれば、プログラムによる実世界の制御はさらに広がると予想される。

また、建築物の構成要素をプログラマブルにする試み\cite{squama}や、
プログラムによってその構成を動的に変化させるモジュールについての研究もなされている。
これらの研究が実用化されていけば、生活空間や物質の制御にもプログラムが利用される。

現在でもプログラムは様々な領域において制御を担っているが、
今後もプログラムが制御可能な領域は広がっていくと考えられる。

\section{あらゆる処理をプログラムで記述する}\label{ux3042ux3089ux3086ux308bux51e6ux7406ux3092ux30d7ux30edux30b0ux30e9ux30e0ux3067ux8a18ux8ff0ux3059ux308b}

第\ref{sec:human-as-computational-resources}節から、プログラムにおいて人間と計算機は
同じ計算資源として振る舞うようになっていくと考えられる。
また、第\ref{sec:area-of-program}節のように、プログラムが制御を担う領域は今後も広がっていくことが予想される。
プログラムによって実世界の要素を制御することは日常的になっていくことは確実である。
これらのことから、プログラムは処理する領域を限定することなく、処理を実行する対象も限定することはなくなっていくと考えられる。
つまり、プログラムで人間と計算機の双方を用いて実世界・電子世界の双方を処理することのできる環境ができあがるということである。

このような環境が一般的になれば、例えば、人間を用いて実世界におけるタスクを実行するプログラムが実現する。
つまり、プログラムに人間の振る舞いや行動を組み込むことが可能となる。
上記のような統合を実現するために注目すべきなのが、マニュアルやレシピといったものだ。
マニュアルやレシピは構造的に書かれた人間への指示書であり、プログラムとの類似性が高い。
料理レシピは調理する上で必要となる処理が記述されたものである。
コンピュータがプログラムを元に処理を実行することと同じように、
人間はレシピを元に調理という名の処理を実行する。
料理レシピの一部をプログラムとして表現すると、ソースコード\ref{code:background-cooking}のようになる。

\begin{lstlisting}[caption=料理レシピの一部の擬似コードで表す, label=code:background-cooking]
if 鍋の水が沸騰する() == true
  パスタを鍋に投入する()
else
  待つ()
\end{lstlisting}

また、仕事のマニュアルも同様である。
その時々で行うべき業務はマニュアルなどによって定義されており、人間はマニュアルを見たり、覚えておくことによって
処理を実行する。
例えば、小売店の店員の仕事はソースコード\ref{code:background-retail}のようなプログラムとして表現可能である。

\begin{lstlisting}[caption=小売店の店員の挙動の一部を擬似コードで表す, label=code:background-retail]
if レジに人が並んでいる() == true
  2番レジを開ける()
else
  品出しをする()
\end{lstlisting}

レシピやマニュアルといったように、人間が実行すべき処理を記述してある手順書は多く存在する。
これは、ただ実行する対象が人間なだけであって、プログラムと同類のものであると言える。
人間にとっての処理もプログラムのような記述方法で表現可能であることから、
今まで実世界で人間が行っていたようなタスクでもプログラムに変換可能と言える。

実世界におけるタスクをプログラミングすることによって、様々なメリットが生まれる。
例えば、処理に応じて人間と計算機を切り替えて実行させ、分業させることが可能となる。
計算機が得意なことは計算機が、人が得意なことは人が実行するようにすれば、より効率的であったり、
確実性の高い処理が実現する。
可能な限り計算機に処理を実行させることで、人間の負担を減らすことも可能だ。

しかし、現状の仕組みではプログラム上で人間を用いて実世界のタスクの処理を実現する環境は十分に整っていない。
人間を計算資源としたシステムの多くはクラウドソーシングのプラットフォームを利用している。
インターネットを介した不特定多数の人間を利用していることから、主に人間の知能しか利用できず、
実世界のタスクを実行してもらうことは困難だ。
例えば、「料理をする」といったプログラムは、インターネットを介した他人に実行してもらっても無意味である。
自分の日常的な活動をプログラミングしたいという時であれば、自分自身を実行対象に指定することが出来なければならない。
また、家庭内での活動をプログラミングしたい場合でも、家族だけを実行対象に指定したい。

人間が優れているのは知能だけではない。
人間の身体や五感を使った処理には、現在の計算機やセンサー・アクチュエータでは実現が困難な処理が多く存在する。
人間の能力の一部だけしか利用できない既存のクラウドソーシングベースの仕組みでは、
人間の能力を最大限に活用した人力処理を実現しているとは言えない。

そこで本研究では、人間と計算機の処理を融合させることを目的とした新しいプログラミング環境を提案する。
提案するプログラミング環境は、以下のような特徴を持つ。

\begin{itemize}
\itemsep1pt\parskip0pt\parsep0pt
\item
  人間と計算機への指示を類似の記法で実現
\item
  実行者を具体的に指定可能
\item
  実世界のタスクを処理することを考慮した実装
\end{itemize}

本研究で提案するプログラミング環境によって、新しいプログラミングの可能性を追求する。

\section{まとめ}\label{ux307eux3068ux3081}

本章では、プログラムの現状について示し、人間とコンピュータが対等な処理実行のリソースであるということを示した。
また、プログラムが実行対象を選ばない汎用的な処理記述フォーマットであることを示し、
あらゆる処理を記述するためのプログラミング環境の実現に向けた状況を考察した。
