\chapter{背景}\label{chap:background}

本章では、人間が計算資源としてプログラムから利用されようになっている現状について述べ、
人間と計算機への指示を融合させるプログラミング環境について説明する。

\newpage

\section{計算資源としての人間}\label{sec:human-as-computational-resources}

コンピュータのみでは解決が困難な問題を、人間を計算資源として利用することによって解決する考え方は
ヒューマンコンピュテーション\cite{humancomputation}と呼ばれる。
コンピュータは十分に優れた処理能力を有するが、パターン認識など人間のほうが得意な処理分野は多く存在する。
人間と計算機のそれぞれにお互いの得意な領域を割り振ることで、
コンピュータプログラムのみで完結することが困難であったような問題でも処理できるようになった。

ヒューマンコンピュテーションの例として挙げられるシステムに、reCAPTCHA\cite{recaptcha}がある。
reCAPTCHAは、人間かコンピュータを識別するためのテストとして活用されているCAPTCHA\cite{captcha}を応用したものである。
reCAPTCHAでは、人間かコンピュータかの識別を行いながら、
コンピュータでは識別が困難な文字の認識を人間に実行させている。

他の例として、Soylent\cite{soylent}といったソフトウェアも存在する。
Soylentは、文章の校正作業をインターネットを介して他の人間たちに実行させるシステムである。
文章の校正も人間のほうが得意であったり、内容によっては人間にしかできない作業である。
この他にも、多くのヒューマンコンピュテーションの考えを活用したシステムが存在する。

インターネットを介して不特定多数の人間に仕事を依頼する仕組みはクラウドソーシング\cite{riseofcrowdsourcing}と呼ばれる。
必要なときに必要なだけの人材を安価に集めることができ、近年注目を浴びている。
クラウドソーシングにおいてもヒューマンコンピュテーションの考えが反映された事例は多く存在し、
多数の人間を計算資源とした処理が実現されている。

近年ではスマートフォンやタブレットデバイスの普及によって、人間はインターネットを介していつでもあらゆる場所で、
様々なシステムに接続可能となっている。
これは、利用者側にとっては便利にインターネットを介して情報を得られるようになったということであり、
システム側にとってはいつでも利用者とコミュニケーションが取れるようになったということでもある。
また、これらのデバイスは気軽に持ち運ぶことができるため、家の中や外出先等、どこでもやりとりができるようになっている。
つまり、いつでもどこでも、デバイスを介することで、人間を計算資源として利用する下地が出来つつあるということである。
人間は今まで以上に計算資源として利用されるようになると考えられる。

ヒューマンコンピュテーションやクラウドソーシングの流行によって、
人間を優秀な計算資源として捉え活用する事例は増えている。
人間と計算機、どちらが実行するとしても、実現したい処理が可能で結果を得ることが出来れば問題はない。
ある処理を実現する上では、人間も計算機も処理を実行するための同じ資源であり、
計算機が得意なことは計算機が、人間が得意なことは人間が実行すれば良い。
人間と計算機は計算資源として等価になっていき、その境界が曖昧になっていくことが予想される。

\section{あらゆる処理をプログラムで記述する}\label{ux3042ux3089ux3086ux308bux51e6ux7406ux3092ux30d7ux30edux30b0ux30e9ux30e0ux3067ux8a18ux8ff0ux3059ux308b}

人間を計算資源とすることで、実世界での人間の振る舞いや行動を取り入れたプログラムを記述し実行することができる。
計算資源という言葉は、CPUやメモリ等、演算を主に担うものだけではなく、入出力装置等もその範疇に入る。
人間が持つ優れた機能は知能だけではない。
身体という非常に優れた機能も持っており、知能と身体を組み合わせた優秀な入出力装置としても利用可能である。

人間の振る舞いや行動を部分的に取り入れ統合を実現している例として注目すべきものが、マニュアルやレシピである。
マニュアルやレシピは人間に対する指示書であり、プログラムの一種である。
例えば、料理レシピは調理をする上で必要となる処理が記述されたプログラムである。
計算機がプログラムを元に処理を実行することと同じように、
人間はレシピを元に調理という名の処理を実行する。
料理レシピの一部を擬似コードとして表現すると、ソースコード\ref{code:background-cooking}のようになる。

\begin{lstlisting}[caption=料理レシピの一部を擬似コードで表す, label=code:background-cooking]
if 鍋の水が沸騰している? == true
  パスタを鍋に投入する()
else
  待つ()
\end{lstlisting}

また、仕事のマニュアルも同様である。
マニュアルはその時々で行うべき業務を定義したものであり、人間はマニュアルを見たり覚えておくことによって、処理を実行する。
例えば、小売店の店員の仕事はソースコード\ref{code:background-retail}のような擬似コードとして表現可能である。

\begin{lstlisting}[caption=小売店の店員の挙動の一部を擬似コードで表す, label=code:background-retail]
if レジに人が並んでいる? == true
  2番レジを開ける()
else
  品出しをする()
\end{lstlisting}

レシピやマニュアルといったように、人間が実行すべき処理を記述している手順書は多く存在する。
これは、ただ実行する対象が人間なだけであって、コンピュータプログラムと同種のものである。
人間の振る舞いや行動もプログラムのような記述方法で表現可能であることから、
今まで実世界で人間が行っていたような処理でもプログラムに変換可能と言える。

実世界における処理をプログラミングすることによって、様々なメリットが生まれる。
例えば、処理に応じて人間と計算機を切り替えて実行させ、分業させることが可能となる。
計算機が得意なことは計算機が、人間が得意なことは人間が実行するようにすれば、
より効率的であったり、確実性の高い処理が実現する。
可能な限り計算機に処理を実行させることで、人間の負担を減らすことも可能である。

しかし、現状の仕組みでは人間を用いた実世界における処理のプログラミングを実現する環境は十分に整っていない。
人間を計算資源としたシステムの多くはクラウドソーシングのプラットフォームを利用している。
インターネットを介した不特定多数の人間を利用していることから、主に人間の知能しか利用できず、
実際に手を動かして実世界における処理を実行してもらうことは困難である。
例えば、「料理をする」といったプログラムは、インターネットを介した他人に実行してもらっても無意味である。
自分の日常的な活動をプログラミングしたいという時であれば、自分自身を実行対象に指定することが出来なければならない。
また、家庭内での活動をプログラミングしたい場合でも、家族だけを実行対象に指定したい。

人間が優れているのは知能だけではない。
人間の身体を利用することで可能な処理には、
現在の計算機やセンサー・アクチュエータ技術では実現が困難な処理が多く存在する。
これらの要素を複合的に利用することで、人間の能力を最大限に活用できる。
人間の能力の一部だけしか利用できないクラウドソーシングベースの仕組みでは、
人間の能力を最大限に活用した人力処理を実現しているとは言えない。

そこで本研究では、人間と計算機の処理を融合させることを目的とした新しいプログラミング環境を提案する。

提案するプログラミング環境は、以下のような特徴を持つ。

\begin{itemize}
\itemsep1pt\parskip0pt\parsep0pt
\item
  人間と計算機への指示を類似の記法で実現
\item
  実行者を具体的に指定可能
\item
  実世界における処理の実行を考慮した実装
\end{itemize}

本研究で提案するプログラミング環境によって、新しいプログラミングの可能性を追求する。

\section{まとめ}\label{ux307eux3068ux3081}

本章では、計算資源としての人間の現状について整理し、人間と計算機が対等な処理実行のリソースであるということを示した。
また、プログラムが実行対象を選ばない汎用的な処理記述フォーマットであることを示し、
あらゆる処理を記述するためのプログラミング環境の実現に向けた状況を考察した。
