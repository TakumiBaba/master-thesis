\chapter{結論}\label{chap:conclusion}

本章では、本論文を統括する。

\section{論文の統括と結論}\label{ux8ad6ux6587ux306eux7d71ux62ecux3068ux7d50ux8ad6}

本研究の成果を以下にまとめる。

\begin{itemize}
\itemsep1pt\parskip0pt\parsep0pt
\item
  考える
\item
  人間と計算機の処理を融合させたプログラミング環境の設計と実装を行った。
\item
  考える
\end{itemize}

プログラムは今後さらにその処理領域を広め、社会において必要とされるものとなっていくと考えられる。
世の中の多くがプログラムによって制御される。
また、人間も計算資源としてシステムに組み込まれ、人間と計算機が共同して様々な処理を実現していくと予想される。

しかし、既存の仕組みでは、人間にしか実行できないような実世界におけるタスクを含んだプログラムを
処理することが困難である。
また、人間と計算機をまったく同じ記法で扱うことはできない。
そこで、自分自身や家族等、身近な人間を計算資源として統合し、普通のプログラミング記法で記述・実行可能な
プログラミング環境について提案した。
この仕組みによって、より簡単に人間をプログラムに組み込んで、実世界におけるタスクを含んだプログラムを
実行することが可能になる。
仕事のプログラム化等、プログラムによって人間の行動をサポートすることができるなど、
有用性のあるプログラミング環境であると言える。

今後は、システムの改良や、考察で述べた点についての改善などを行うことで、
より有用なシステムにしていく。
