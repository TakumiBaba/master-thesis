\chapter{結論}\label{chap:conclusion}

本章では、本論文を統括する。

\newpage

\section{論文の統括と結論}\label{ux8ad6ux6587ux306eux7d71ux62ecux3068ux7d50ux8ad6}

本論文では、人間と計算機の処理を融合させたプログラミング環境の設計と実装を行った。
計算資源としての人間を活用する事例は今後増えていくと考えられる。
人間が様々なシステムに組み込まれ、計算機と協働して様々な処理を実現していくことができれば、
より効率的・正確なシステムが構築可能となるだろう。

既存研究においてもプログラム内で人間を計算資源として利用する研究はなされているが、
それらの研究では人間の知能を利用する程度の統合しか果たされていない。
人間は知能だけが優秀なのではなく、その身体と知能の両方を利用することで最大限の力を発揮すると考えられる。

身体操作の指示まで踏み込んだ人間と計算機への指示の融合を実現させたプログラミング環境があれば、様々なことに活かせる。
そこで、自分自身や家族を対象とするような、特定の人間を計算資源として統合し、従来のプログラミング言語上で記述・実行可能な
プログラミング環境について設計し、実装した。
人間とプログラムのインタラクション設計はシンプルであるが、十分利用可能であると考える。
この仕組みによって人間をプログラムに組み込むことで、
人間の振る舞いや行動を活用した実世界における処理を記述したプログラムを実行することが可能になる。
単純にプログラムで記述出来る領域を広げるだけでなく、
仕事をプログラム化して実行するなど、プログラムによる人間行動のサポートも可能である。

今後は、システムの改良や、考察で述べた点についての改善などを行うことで、
より有用なシステムにしていく。
