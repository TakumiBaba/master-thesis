\chapter{結論}\label{chap:conclusion}

本章では、本論文を統括する。

\section{論文の統括と結論}\label{ux8ad6ux6587ux306eux7d71ux62ecux3068ux7d50ux8ad6}

本論文では、人間と計算機の処理を融合させたプログラミング環境の設計と実装を行った。
また、提案したプログラミング環境における人間の応答速度を明らかにした。

プログラムは今後さらにその処理領域を広め、社会と生活、世界を構成する重要な要素となっていく。
また、人間も計算資源としてシステムに組み込まれ、人間と計算機が協働して様々な処理を実現していくと予想される。

既存研究においても人間を計算資源として利用する研究はなされているが、それらの研究では人間の知能を組み込む
レベルの統合しか果たされていない。
人間は知能だけが優秀なのではなく、その身体と知能の両方を利用することで最大限の力を発揮する。
身体操作の指示まで踏み込んだ人間と計算機への指示の融合を実現させたプログラミング環境が必要である。

そこで、自分自身や家族等、身近な人間を計算資源として統合し、普通のプログラミング記法で記述・実行可能な
プログラミング環境について提案した。
この仕組みによって、より簡単に人間をプログラムに組み込んで、実世界におけるタスクを含んだプログラムを
実行することが可能になる。
仕事のプログラム化等、プログラムによって人間の行動をサポートすることができるなど、
有用性のあるプログラミング環境であると言える。

今後は、システムの改良や、考察で述べた点についての改善などを行うことで、
より有用なシステムにしていく。
