\chapter{Babascriptプログラミング環境の設計}
\label{chap:design}

本章では、前章におけるヒューマンコンピュテーションやクラウドソーシングの研究動向を踏まえ、
人とプログラムとの新しいインタラクションを実現するためのプログラミング環境の要件を定義し、考察を行う。

\section{プログラムと人のインタラクション設計}\label{ux30d7ux30edux30b0ux30e9ux30e0ux3068ux4ebaux306eux30a4ux30f3ux30bfux30e9ux30afux30b7ux30e7ux30f3ux8a2dux8a08}

\subsection{具体的な人間リソースの指定}\label{ux5177ux4f53ux7684ux306aux4ebaux9593ux30eaux30bdux30fcux30b9ux306eux6307ux5b9a}

\subsection{関数のように人間を扱う}\label{ux95a2ux6570ux306eux3088ux3046ux306bux4ebaux9593ux3092ux6271ux3046}

\subsection{類似の指示記法で実現したい}\label{ux985eux4f3cux306eux6307ux793aux8a18ux6cd5ux3067ux5b9fux73feux3057ux305fux3044}

人と計算機の両要素を融合させたプログラムを書くためには、
人と計算機、双方への指示を同じようなモデルで実行できるようにする必要がある。

\section{プラガブルなモジュール構成}\label{ux30d7ux30e9ux30acux30d6ux30ebux306aux30e2ux30b8ux30e5ux30fcux30ebux69cbux6210}

\subsection{通信手法}\label{ux901aux4fe1ux624bux6cd5}

\subsection{ユーザインタフェース}\label{ux30e6ux30fcux30b6ux30a4ux30f3ux30bfux30d5ux30a7ux30fcux30b9}

\begin{itemize}
\item
  デバイスが多様化している
\item
  デバイスに応じて利用可能なリソースが異なる
\item
  例えば以下の要素
\item
  通信手法
\item
  ユーザインタフェース
\item
  簡単に開発できる環境のためにも、これらの独自開発に負担をかけないようにするべき
\item
  プラガブルにする
\item
  通信手法をプラガブル
\item
  ユーザインタフェースをプラガブル
\end{itemize}

\section{拡張性の確保}\label{ux62e1ux5f35ux6027ux306eux78baux4fdd}

人間が時計をつけたりJawboneUpみたいなのをつけるのと同じように、プログラム上の人間もその機能を拡張できるべきだ。

\section{新しいプログラミング環境}\label{ux65b0ux3057ux3044ux30d7ux30edux30b0ux30e9ux30dfux30f3ux30b0ux74b0ux5883}

人間と計算機、双方への指示を同じプログラム

上記項目を満たすアプリケーションが必要 下記の4つを実装した

\begin{itemize}
\itemsep1pt\parskip0pt\parsep0pt
\item
  Babascript
\item
  Babascript Client
\item
  Babascript Plugin
\item
  Babascript
\end{itemize}

\section{まとめ}\label{ux307eux3068ux3081}
