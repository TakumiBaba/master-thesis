\chapter{Babascriptプログラミング環境の設計}
\label{chap:design}

本章では、前章におけるヒューマンコンピュテーションやクラウドソーシングの研究動向を踏まえ、
人とプログラムとの新しいインタラクションを実現するためのプログラミング環境の要件を定義し、考察を行う。

\section{設計要件}\label{ux8a2dux8a08ux8981ux4ef6}

何が必要なのか

\section{プログラムと人のインタラクション設計}\label{ux30d7ux30edux30b0ux30e9ux30e0ux3068ux4ebaux306eux30a4ux30f3ux30bfux30e9ux30afux30b7ux30e7ux30f3ux8a2dux8a08}

\subsection{プログラムと人間のメッセージ交換モデル}\label{ux30d7ux30edux30b0ux30e9ux30e0ux3068ux4ebaux9593ux306eux30e1ux30c3ux30bbux30fcux30b8ux4ea4ux63dbux30e2ux30c7ux30eb}

\begin{itemize}
\itemsep1pt\parskip0pt\parsep0pt
\item
  プログラムの処理の実行対象はコンピュータのみだった
\item
  プログラムを人間にも処理させるためには、人間へのメッセージングの仕組みが必要である
\item
  プログラムはそのままコンピュータに処理命令をメッセージングしてると言える。
\item
  人間に直接メッセージングは現状では無理
\item
  何かのデバイスを仲介することでメッセージングが実現可能である
\item
  このモデルによって、コンピュータと人間は処理実行対象として同じになれる
\end{itemize}

\subsection{具体的な人間リソースの指定}\label{ux5177ux4f53ux7684ux306aux4ebaux9593ux30eaux30bdux30fcux30b9ux306eux6307ux5b9a}

\subsection{類似の指示記法で実現したい}\label{ux985eux4f3cux306eux6307ux793aux8a18ux6cd5ux3067ux5b9fux73feux3057ux305fux3044}

人と計算機の両要素を融合させたプログラムを書くためには、
人と計算機、双方への指示を同じようなモデルで実行できるようにする必要がある。

\section{プラガブルなモジュール構成}\label{ux30d7ux30e9ux30acux30d6ux30ebux306aux30e2ux30b8ux30e5ux30fcux30ebux69cbux6210}

あらゆるデバイスで使えるようにする必要がある。
デバイスは多様化している。 利用可能なリソースが異なる。
そのためには、少しプログラムを書くだけで色々なデバイスで転用可能にしなくてはならない。
そこで、各デバイスごとに仕様が異なることの多い部分について、プラガブルなモジュール構成にすることで
この問題を解決できると考えた。

\subsection{通信}\label{ux901aux4fe1}

通信手法は各デバイスごとに大きくことなる。
出来れば、プログラムが実行され、人間に対するメッセージングが行なわれたらすぐに人にメッセージが届くべきである。
そのため、リアルタイム通信を前提にする必要がある。
例えば、パソコンのwebブラウザ上やサーバ上での実行であれば、通信手法が限られることはあまりない。
自由に通信の手法を選択することが可能である。
しかし、例えばスマートフォンの場合は、OSの仕様上、リアルタイム通信が切断されてしまうようなこともある。

その場の環境的に、通信がしにくいといった場合には、リアルタイム通信を前提とした通信手法は使いづらい。
遅延を前提とするような通信手法を選択できるべきである。

こういったことは、日常活動を送っていればよくあることである。
通信モジュールをプラガブルにし、様々な場面に対応したモジュールを簡単に作れるようにしておく必要がある。

\subsection{ユーザインタフェース}\label{ux30e6ux30fcux30b6ux30a4ux30f3ux30bfux30d5ux30a7ux30fcux30b9}

メッセージの提示や実行結果を入力する画面が必要となるが、このユーザインタフェースはデバイスごとに大きく異なると考えられる。
その要因として、スクリーンサイズの大きさがある。
例えば、比較的画面サイズに余裕のあるノートパソコンやデスクトップパソコンであれば、問題なく提示することができる。
しかし、スマートフォンやスクリーンの付いた腕時計型のデバイスなどであれば、同じユーザインタフェースのまま表示することは難しいだろう。

また、利用可能なアプリケーションによっても異なる。
Webアプリケーションとして実装する場合、Webブラウザが使えればいい。
チャット上に実装したい場合は、ユーザインタフェースが大きく異なる。

\section{拡張性}\label{ux62e1ux5f35ux6027}

人間が時計をつけたりJawboneUpみたいなのをつけるのと同じように、プログラム上の人間もその機能を拡張できるべきだ。

\section{新しいプログラミング環境}\label{ux65b0ux3057ux3044ux30d7ux30edux30b0ux30e9ux30dfux30f3ux30b0ux74b0ux5883}

前節までの設計検討から、人間と計算機への指示を融合させた新しいプログラミング環境の要件についてまとめる。

人間と計算機、双方への指示を同じプログラム

上記項目を満たすアプリケーションが必要 下記の4つを実装した

\begin{itemize}
\itemsep1pt\parskip0pt\parsep0pt
\item
  Babascript
\item
  Babascript Client
\item
  Babascript Plugin
\item
  Babascript
\end{itemize}

\section{まとめ}\label{ux307eux3068ux3081}
