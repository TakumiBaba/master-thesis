\chapter{考察}\label{chap:discussion}

本章では、Babascriptプログラミング環境における諸問題や
可能性について述べる。

\newpage

\section{処理実行単位としての人}\label{ux51e6ux7406ux5b9fux884cux5358ux4f4dux3068ux3057ux3066ux306eux4eba}

本研究において、人間は処理すべきことを指示され実行する存在、コンピュータやセンサーと同じ処理単位としてみなされる。
コンピュータと同じ存在ということに心理的な拒否感を覚えるといったことが考えられる。
しかし、処理を指示され実行する立場に徹するということは、指示されたことのみに集中していれば良いということであり、
非常に楽なことでもある。
また、コンピュータに出来ることはコンピュータにやらせ、人間は人間が得意なことや人間にしか出来ないようなことを実行する
ことになるため、ワーカーにとっては作業量が減るなどのメリットも存在する。
指示される立場でいるということは、作業を楽にするための一つのアイデアであるとも言える。

また、指示に積極的に従うことに酔って、今までは実現出来なかった処理が実現したり、全体的な処理の正確性が向上するとすれば
それはワーカー側にとってもメリットであると言える。

\section{プログラム化のメリット}\label{ux30d7ux30edux30b0ux30e9ux30e0ux5316ux306eux30e1ux30eaux30c3ux30c8}

Babascriptプログラミング環境によって様々な処理をプログラム化した際に得られるメリットについて考察する。

まず、プログラムとして記述しておくことによって、コンピュータと一緒に処理を実行していくことができる。
例えば、普通の計算は明らかにコンピュータのほうが得意だ。
人間のインプット等を元にしてコンピュータに計算させれば、高速かつ正確な結果を得られ、人間も計算しなくて良い。
記憶しておくことも、コンピュータのほうが得意だ。
仕事や実世界の状態をコンピュータで管理することによって、時間が経ったりしても正確に情報を引き出すことが出来る。

プログラムとして記述することで、様々な情報を取得が容易になる。
Aという処理にかかった時間は、少し前のプログラムの場合は1分であったが、改善することで30秒にすることができた、
というようなことは、解析プログラムを導入することですぐにわかるだろう。
AさんとBさんが同じ処理を実行した場合の実行時間の差などもデータとして取得することが容易となる。
改善の結果が悪ければすぐに元のプログラムに戻す、といったことも、バージョン管理が一般的に行なわれている
プログラムだからこそ、実行しやすい。

また、複数の人間をオーケストレーションし、効率的に動かすといったことも可能になる。
マニュアルなどでは、基本的に一人の人間を対象としている。
しかし、プログラムで記述し、各自に指示を送るようにすれば、複数の人間の指揮も容易であると考えられる。

\section{型指定の効果}\label{ux578bux6307ux5b9aux306eux52b9ux679c}

ユーザ評価実験によって、返り値の型の違いが人の処理の実行速度に関係してくることがわかった。
これは、入力にかかるコストが型によって異なるということが大きな要員と考えられる。
Babascriptプログラミング環境においては、人間への指示に返り値の型を指定することができる
ユーザはその型に合ったインタフェースが提示される。
そのため、キーボードによる入力等が必要になる返り値の型の場合、実行速度が遅くなる。
String型の場合、その問題が顕著であり、ユーザの文字入力の速度に依存する。
速度を重視する場合は、Boolean型を利用するのが一番高速であるが、返答がtrueかfalseの2択となるため、
より指示内容を具体的にする必要がある。

また、型を指定されることによって、ユーザ側も入力すべき値がはっきりとし、迷うことがなかったという意見もあった。
一方で、String型の場合、指示内容にもよるが入力の自由度が高くなりがちという問題がある。
そのため、具体的にどんな値を入力すれば良いのかわからなくなった、という意見があった。
これは、どんな値を入力すべきかという例示をすることによって、ある程度解決可能な問題だと考えられる。

\section{処理の実行遅延と実行保証}\label{ux51e6ux7406ux306eux5b9fux884cux9045ux5ef6ux3068ux5b9fux884cux4fddux8a3c}

Babascriptを用いて人に指示を送っても、指示を受け取った人がその場ですぐに処理を実行するとは限らず、
遅延して実行される可能性がある。
指示を受け取るデバイスを見ていないという場合や、指示を受け取っても状況的にすぐに実行できないという場合があると
考えられ、その場合はすぐに指示に対して実行結果を返すことが出来ない。
Babascriptでは、こういった状況に対応するため、非同期実行を前提とした設計にしている。
そのため、人間による処理の実行中はコンピュータ側に他の処理を実行させておくといったことも可能だが、
人間による処理はほぼ確実に全体の処理を遅延させると考えられる。

また、指示を無視するといったことが起こりうる。
つまり、実行を完全に保証することができない。
上司からの命令などのように、労働上のある程度の強制力がある場合や、
自分自身への指示、家族からの指示などの場合は、無視の可能性は低くなるが、
強制力がない場合は、指示を無視するといったことが起こりうる。
こういった状況においては、何かしらのインセンティブをワーカーに与えることによって実行保障性を確保することが
できると考えられる。

\section{エラーへの対応}\label{ux30a8ux30e9ux30fcux3078ux306eux5bfeux5fdc}

Babascriptからの指示を受け取った時、その指示が想定している状況と、現実の状況が大きく乖離している可能性がある。
場所や時間に依存するような指示の場合、特に発生することが予想される。
この場合、指定された型では適切な値を返せない可能性が存在する。
また、無理やり値を返そうとした結果、本来返されるべき値をは異なる値が処理結果として入力されてしまう可能性も存在する。
指示が想定している環境と現実の環境が明らかに異なっている場合などには、エラーをプログラムに通知できる必要がある。

また、エラーでプログラムが終了した場合、途中から実行しなおすといったことが現状では出来ない。
実世界におけるタスクの多くは、途中で間違っても、その間違いがタスク継続が困難なものでない限り、途中からやり直すといったことが可能だ。
Babascriptで実現するプログラムにおいても、途中からやり直すという機構が求められる。
Turkit\ref{turkit}におけるcrash-and-rerun
programming等の概念のように、途中経過を保存しておくことで、
プログラムの途中から実行し直すという仕組みが実現可能であると考える。
また、途中経過保存によって、どんなタスクがエラーを起こしやすいのかといったことも保存することができ、
ワークフロー改善にも役立つと考えられる。

\section{処理の先送り}\label{ux51e6ux7406ux306eux5148ux9001ux308a}

プログラムから指示を受け取った時、今でなくても後でなら処理できるという状況が存在する。
しかし、そのまま指示を放置していては、タイムアウト処理等によって指示が撤回されてしまったり、
他の処理が遅延してしまう可能性が高い。
処理の先送りをプログラムに通知しておき、またあとで指示を通知し直すなどの仕組みを用意する必要がある。

これは、Babascirpt
Agent側にも非同期的に指示を通知するインタフェースが必要となる、ということである。
現状では、Babascript Agentは完全に同期的な挙動をする。
先送りした指示や現時点でキューに溜まっている指示などをリスト型のインタフェースで表示しておき、
実行する処理を選べるような設計をすることで、Babascript
Agent側で処理の先送りを実現可能である。

\section{割り込み処理}\label{ux5272ux308aux8fbcux307fux51e6ux7406}

コンピュータが多くの割り込み処理を行っているのと同じように、
様々な指示が送られる中で、人間に対しても、指示を割り込みできるような仕組みが求められる。
例えば、現実においても、割り込み処理は多くなされている。
料理をしてる時に、鍋が吹きこぼれそうだったら、他の処理を中断してでも鍋から吹きこぼれないような処理を行う。

このような割り込み処理を、Babascriptでは、人間の指示構文実行時のオプション情報として
特別なフラグを立てることで実現させている。
人間への指示は、Node-Linda上にキュー形式で保存されているが、
この指示のキューを通常タスクと優先タスクで分けて保存している。
優先タスクキューに指示が入ってる場合は、こちらのキューから優先して指示を取得する。

割り込み処理に関しては、インタフェース側の工夫も重要となってくる。
現状では、高度な工夫は実現できていあに。
割り込み処理であるということを明示的にし、割り込んだ処理をすぐに行うように
誘導させられるようにインタフェースを再実装する必要がある。

\section{命令の抽象度設計の必要性}\label{ux547dux4ee4ux306eux62bdux8c61ux5ea6ux8a2dux8a08ux306eux5fc5ux8981ux6027}

Babascriptでは、人への指示構文の記述には制限がない。
そのため、命令の抽象度はプログラマの記述能力に依存する。
抽象度が高すぎる指示内容にしてしまうと、ワーカーにとって実行内容が理解しづらくなってしまう。
結果として、想定外の処理が実行されたり、意図しない処理結果を返される可能性が存在する。
人間は柔軟な解釈が可能なため、ある程度は補正可能だと考えられるが、補正が不可能なほど抽象的な
指示内容の場合、問題が発生する。

具体的過ぎる命令は、全体の処理内容にもよるが、プログラム自体が冗長となり得る。
プログラムとワーカーの間のやりとりが増え、通信や待機時間、入力時間などによって処理全体が遅延すると考えられる。
また、ワーカーにとっても、やりとりが増えることで、入力の手間が増え、負担増に繋がる。

指示ごとに異なると考えられるが、命令文は適切な抽象度に設計しなくてはならない。

\section{異なる種類の指示の複数同時実行}\label{ux7570ux306aux308bux7a2eux985eux306eux6307ux793aux306eux8907ux6570ux540cux6642ux5b9fux884c}

複数のプログラムから同時に一人の人間に複数の関係のない指示が実行される可能性がある。
例えば、料理プログラムと掃除プログラムが同時に実行された場合、
鍋で何かを煮ている途中に、「洗剤を投入する」といった指示がなされるといったことが考えられる。

現実世界で同じようなことが起きても、人間は各動作を別スレッドで動作させることで問題を回避する。
Babascript Agent
Applicationにおいても、同様の手法を取ることで回避が可能であると考える。
どのプログラムからの指示のなのかを明示したり、今までの指示内容を示すといった手法を取ることによって
解決可能である。 この手法の場合、既存の実装では対応できていない。

\section{Babascript
Agentへのキャラクター性の付与}\label{babascript-agentux3078ux306eux30adux30e3ux30e9ux30afux30bfux30fcux6027ux306eux4ed8ux4e0e}

Babascript
Agentは、一種のヒューマンエージェントインタラクションであると言える。
ヒューマンエージェントインタラクションとは、
人間とソフトウェアエージェントやロボット間のインタラクション
設計に関する研究分野である。
ヒューマンエージェントインタラクション研究の分野においては、ソフトウェアエージェントやロボットに対して
擬人化手法を適応させることによって、エージェントをより人間らしく見せたり、コミュニケーションを取りやすく
するといった研究がなされている。
擬人化エージェントがもたらす効果については、村上らによる研究\cite{murakami}では、エージェントに人間関係を適応させた実験を行っている。
村上らの実験では、研究室の教官と学生の関係を利用し、教授の顔を模したキャラクターと関係のないキャラクター、両者からの
依頼に対して学生の行動が変化するかどうかを観察した。
実験の結果、教官キャラクターのほうがより依頼を受理されやすいという結果がわかった。
この結果から、エージェントに人間関係、特に上司などを模したキャラクターをユーザに提示することである程度、
依頼に対してポジティブにとらえてもらうことが可能だと考えられる。

また、好きなキャラクターや俳優などをキャラクターとして付与することも依頼をポジティブに捉え、実行してもらえる
ことに繋げられるのではないかと考えられる、気がする。ここなおす。根拠しめす。ダメなら消す。

これらの擬人化エージェント研究のように、Babascript Agent
Applicationに対しても擬人化手法を取り入れることで
人間が自然に値を返しやすくなるようにできるのではないかと考えられる。
また、キャラクター性の付与によって、プログラムからの指示であるという印象が薄れることが想定され、
指示に対する拒否感も薄れる等の可能性が存在する。

\section{今後の展望}\label{ux4ecaux5f8cux306eux5c55ux671b}

今後は、本章で述べた点についての改善を主に行う。
まず、割り込み処理や処理の先送りなど、より複雑なプログラムを作っていく上で必要となるであろう仕組みを優先して実装していく。
例外処理機構についても、実験等を通して適切な仕組みについて考察を行い、実装を行っていく。
Babascript
Agentへのキャラクター性の付与に関しては、実装の上、実験を行い、その効果を検証していきたい。
