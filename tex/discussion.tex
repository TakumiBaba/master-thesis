\chapter{考察}\label{chap:discussion}

\section{処理への人の積極的な貢献}\label{ux51e6ux7406ux3078ux306eux4ebaux306eux7a4dux6975ux7684ux306aux8ca2ux732e}

\section{処理単位としての人}\label{ux51e6ux7406ux5358ux4f4dux3068ux3057ux3066ux306eux4eba}

本研究において、人間は処理すべきことを指示され実行する存在、コンピュータやセンサーと同じ処理単位としてみなされる。
コンピュータと同じ存在ということに心理的な拒否感を覚えるといったことが考えられる。

本研究では、人は処理を命令され実行する存在となり、プログラム上においてコンピュータやセンサーと同等となる。
こういったことに対して心理的な拒否感を覚えることも考えられる。
しかし、一処理単位として扱うことは、大きなメリットでもある。
命令を実行するだけのノードであるということは、ただ処理内容を実行することにのみ集中すれば良いということだ。
ただやるべきことだけが提示され、その通りに動けば良いということは、深く考える必要がなく、楽であるということが考えられる。
もちろん、全ての処理がただ提示する通りに動けば良いものではないと考えられるが、
作業を楽にするための一つのアイデアであると言える。

\section{プログラムと身近な人の違い}\label{ux30d7ux30edux30b0ux30e9ux30e0ux3068ux8eabux8fd1ux306aux4ebaux306eux9055ux3044}

プログラムからの指示でも、身近な人からの指示でも、たいして変わらないのでは
指示元を隠蔽するような仕組みがあれば、違和感なくできるのでは?

\section{プログラムとマニュアルの違い}\label{ux30d7ux30edux30b0ux30e9ux30e0ux3068ux30deux30cbux30e5ux30a2ux30ebux306eux9055ux3044}

\section{型指定}\label{ux578bux6307ux5b9a}

ユーザ評価実験によって、返り値の型の違いが人の処理の実行速度に関係してくることが証明された。
Babascriptプログラミング環境においては、人間への指示に返り値の型を指定することができる
ユーザはその型に合ったインタフェースが提示される。
そのため、キーボードによる入力等が必要になる返り値の型の場合、実行速度が遅くなる。
String型の場合、その問題が顕著であり、ユーザの文字入力の速度に依存する。
速度を重視する場合は、Boolean型を利用するのが一番高速であるが、返答がtrueかfalseの2択となるため、
より指示内容を具体的にする必要がある。

また、型を指定されることによって、ユーザ側も入力すべき値がはっきりとし、迷うことがなかったという意見もあった。
一方で、String型の場合、指示内容にもよるが入力の自由度が高くなりがちという問題がある。
そのため、具体的にどんな値を入力すれば良いのかわからなくなった、という意見があった。
これは、どんな値を入力すべきかという例示をすることによって、ある程度解決可能な問題だと考えられる。

\section{処理の実行遅延と実行保証}\label{ux51e6ux7406ux306eux5b9fux884cux9045ux5ef6ux3068ux5b9fux884cux4fddux8a3c}

Babascriptによる人への命令構文が実行されても、命令を受け取った人がすぐに命令に対する処理を実行し
値を返すことを完全に保証することはできない。
タスク受信端末を見ていない、受信しても実行できないといった状況の場合、すぐに値を返すことはできない。
Babascriptによる人への命令構文は全て非同期実行する実装となっているため、人力処理の待ち時間に他の処理を
実行させておくことが可能であるが、人力処理の実行待ちが全体の処理を遅延させる可能性は存在する。

また、労働関係にあるなど、強制力がある場合命令は実行されると考えられるが、
強制力がない場合はそもそもタスクを無視するといったことも考えられる。
タスク実行に強制力がない場合は、金銭などのインセンティブを与えるといった手段によって、
実行保障性を確保するといったことが考えられる。

\section{例外処理機構}\label{ux4f8bux5916ux51e6ux7406ux6a5fux69cb}

Babascriptにおいて、命令が想定する状況と現実の状況との乖離によって適切な返り値を選択・記述ができなくなるといった可能性がある。
これは、現実が刻々と変化していることなどから、完全に避ける事の出来ない問題であると考える。
この際、無理やり値を返すといった処理をしてしまうと、本来の状態とは違った判断がなされてしまう危険性がある。
命令文とは明らかに現実が異なっている場合などは、タスク実行者から例外としてプログラムに通知出来るような仕組みの実装によって、問題の解決へと繋げられると考える。

\section{命令の先送り}\label{ux547dux4ee4ux306eux5148ux9001ux308a}

人間とコンピュータの違いに、タスクの管理手法が異なるという点がある。
コンピュータはプログラムの通りに処理を管理し、実行していくが、人間は違う。
気分であったり、状況や環境に応じてタスクの処理手順を前後させるといったことを行っている。
こういった人間の特性についても、仕組みとして取り入れる必要がある。

人間は非同期的存在である。
そのため、各種命令をあとで処理する、といったことができないといけない。

\section{命令の抽象度設計の必要性}\label{ux547dux4ee4ux306eux62bdux8c61ux5ea6ux8a2dux8a08ux306eux5fc5ux8981ux6027}

Babascriptでは、人への指示構文の記述には制限がない。
そのため、命令の抽象度はプログラマの記述能力に依存する。
抽象度が高すぎる指示内容にしてしまうと、ワーカーにとって実行内容が理解しづらくなってしまう。
結果として、想定外の処理が実行されたり、意図しない処理結果を返される可能性が存在する。
人間は柔軟な解釈が可能なため、ある程度は補正可能だと考えられるが、補正が不可能なほど抽象的な
指示内容の場合、問題が発生する。

具体的過ぎる命令は、全体の処理内容にもよるが、プログラム自体が冗長となり得る。
プログラムとワーカーの間のやりとりが増え、通信や待機時間、入力時間などによって処理全体が遅延すると考えられる。
また、ワーカーにとっても、やりとりが増えることで、入力の手間が増え、負担増に繋がる。

指示ごとに異なると考えられるが、命令文は適切な抽象度に設計しなくてはならない。

\section{複数指示の同時実行}\label{ux8907ux6570ux6307ux793aux306eux540cux6642ux5b9fux884c}

複数のプログラムから同時に一人の人間に複数の関係のない指示が実行される可能性がある。
例えば、料理プログラムと掃除プログラムが同時に実行された場合、
鍋で何かを煮ている途中に、「洗剤を投入する」といった指示がなされるといったことが考えられる。

現実世界で同じようなことが起きても、人間は各動作を別スレッドで動作させることで問題を回避する。
Babascript Agent
Applicationにおいても、同様の手法を取ることで回避が可能であると考える。
どのプログラムからの指示のなのかを明示したり、今までの指示内容を示すといった手法を取ることによって
解決可能である。 この手法の場合、既存の実装では対応できていない。

\section{Babascript
Agentへのキャラクター性の付与}\label{babascript-agentux3078ux306eux30adux30e3ux30e9ux30afux30bfux30fcux6027ux306eux4ed8ux4e0e}

Babascript
Agentは、一種のヒューマンエージェントインタラクションであると言える。
ヒューマンエージェントインタラクションとは、
人間とソフトウェアエージェントやロボット間のインタラクション
設計に関する研究分野である。
ヒューマンエージェントインタラクション研究の分野においては、ソフトウェアエージェントやロボットに対して
擬人化手法を適応させることによって、エージェントをより人間らしく見せたり、コミュニケーションを取りやすく
するといった研究がなされている。
擬人化エージェントがもたらす効果については、村上らによる研究\cite{murakami}では、エージェントに人間関係を適応させた実験を行っている。
村上らの実験では、研究室の教官と学生の関係を利用し、教授の顔を模したキャラクターと関係のないキャラクター、両者からの
依頼に対して学生の行動が変化するかどうかを観察した。
実験の結果、教官キャラクターのほうがより依頼を受理されやすいという結果がわかった。
この結果から、エージェントに人間関係、特に上司などを模したキャラクターをユーザに提示することである程度、
依頼に対してポジティブにとらえてもらうことが可能だと考えられる。

また、好きなキャラクターや俳優などをキャラクターとして付与することも依頼をポジティブに捉え、実行してもらえる
ことに繋げられるのではないかと考えられる、気がする。ここなおす。

これらの擬人化エージェント研究のように、Babascript Agent
Applicationに対しても擬人化手法を取り入れることで
人間が自然に値を返しやすくなるようにできるのではないかと考えられる。

\section{今後の展望}\label{ux4ecaux5f8cux306eux5c55ux671b}
