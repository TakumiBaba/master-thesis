\chapter{考察}\label{chap:discussion}

本章では、Babascriptプログラミング環境における諸問題や
可能性について述べる。

\section{ユーザインタビュー}\label{ux30e6ux30fcux30b6ux30a4ux30f3ux30bfux30d3ux30e5ux30fc}

システムの改善や利用者側の心理の確認のため、 7名の被験者(男性7名,
平均23歳)に試用してもらった上でインタビューを行った。
今回のユーザインタビューでは、現状において一番改善点が多いと思われたBabascript
Agentに その議論を絞り、インタビューを行った。

インタビューの内容は表\ref{table:interview}の通りである。

\begin{longtable}[c]{@{}lc@{}}
\caption{インタビュー内容 \label{table:interview}}\tabularnewline
\toprule
& 質問内容\tabularnewline
\midrule
\endfirsthead
\toprule
& 質問内容\tabularnewline
\midrule
\endhead
問1 &
コンピュータと同じ、処理を実行する立場と言われた時、どう思いますか\tabularnewline
問2 &
自分や他人の感性判断をプログラムに組み込むことについて、どう思いますか\tabularnewline
問3 & 返り値のフォーマット指定は 必要でしたか? \tabularnewline
& 考えの幅が狭まるといったことはありますか? \tabularnewline
& 扱いづらいものはありましたか?\tabularnewline
問4 &
好きなキャラや俳優などからの指示のように見えるインタフェースだった場合、\tabularnewline
& 動きに変化があると思いますか?\tabularnewline
問5 &
命令元の人によって、指示に従うかどうかは変わりますか?\tabularnewline
\bottomrule
\end{longtable}

\subsection{コンピュータと人間が対等な立場であることについて}\label{ux30b3ux30f3ux30d4ux30e5ux30fcux30bfux3068ux4ebaux9593ux304cux5bfeux7b49ux306aux7acbux5834ux3067ux3042ux308bux3053ux3068ux306bux3064ux3044ux3066}

プログラムからの指示を受けるという、コンピュータと人間がまったく同じ立場になるということに関して、
拒否感を覚える人がいるのではないかという仮説を確認するために質問をした。

結果として、被験者全員が特に拒否感を覚えることはないと答えた。
また、プログラムから指示を受けているという印象を受けなかったとの意見を得ることができた。

インタフェースをより洗練させることで、プログラムから指示を受けているという印象を薄めていくことによって、
一般の人にも受け入れてもらえるようにすることは可能と考えられる。

\subsection{自分や他人をプログラムに組み込むこと}\label{ux81eaux5206ux3084ux4ed6ux4ebaux3092ux30d7ux30edux30b0ux30e9ux30e0ux306bux7d44ux307fux8fbcux3080ux3053ux3068}

\subsection{返り値の型について}\label{ux8fd4ux308aux5024ux306eux578bux306bux3064ux3044ux3066}

返り値の型指定について、どう感じたかを質問した。
返り値によってインタフェースを制限することによって、ユーザが返しやすくなるのではないかという仮説はどうか

\subsection{Babascript Agent
について}\label{babascript-agent-ux306bux3064ux3044ux3066}

Babascript
Agentの現在のインタフェースはただ指示を表示し、処理結果を入力させるためだけの機能に絞られている。

\subsection{命令元の提示について}\label{ux547dux4ee4ux5143ux306eux63d0ux793aux306bux3064ux3044ux3066}

Babascript Agent
の現在のインタフェースでは、誰が指示を送っているのかというのがわからない。
命令元がまったくわからない状況において指示に従うことに関してどう感じているのかを聞いた。

\subsection{その他}\label{ux305dux306eux4ed6}

\section{処理実行単位としての人}\label{ux51e6ux7406ux5b9fux884cux5358ux4f4dux3068ux3057ux3066ux306eux4eba}

本研究において、人間は処理すべきことを指示され実行する存在、コンピュータやセンサーと同じ処理単位としてみなされる。
ユーザインタビューからコンピュータサイエンスに関する研究を行っている大学生及び大学院生に関しては、
特に抵抗がないということがわかったが、コンピュータサイエンスの知見の少ない一般の人にとっては、
コンピュータと同じ存在ということに心理的な拒否感を覚えるといったことが考えられる。
しかし、処理を指示され実行する立場に徹するということは、指示されたことのみに集中していれば良いということであり、
非常に楽なことでもある。
また、コンピュータに出来ることはコンピュータにやらせ、人間は人間が得意なことや人間にしか出来ないようなことを実行する
ことになるため、ワーカーにとっては作業量が減るなどのメリットも存在する。
指示される立場でいるということは、作業を楽にするための一つのアイデアであるとも言える。

また、指示に積極的に従うことによって、今までは実現出来なかった処理が実現したり、全体的な処理の正確性が向上するとすれば
それはワーカー側にとってもメリットであると言える。

\section{プログラム化のメリット}\label{ux30d7ux30edux30b0ux30e9ux30e0ux5316ux306eux30e1ux30eaux30c3ux30c8}

Babascriptプログラミング環境によって様々な処理をプログラム化した際に得られるメリットについて考察する。

まず、プログラムとして記述しておくことによって、コンピュータと一緒に処理を実行していくことができる。
例えば、普通の計算は明らかにコンピュータのほうが得意だ。
人間のインプット等を元にしてコンピュータに計算させれば、高速かつ正確な結果を得られ、人間も計算しなくて良い。
記憶しておくことも、コンピュータのほうが得意だ。
仕事や実世界の状態をコンピュータで管理することによって、時間が経ったりしても正確に情報を引き出すことが出来る。
一方で、あいまいな出来事の処理や感性的な判断を求められる場面、実世界を対象とした処理などは
人間のほうが得意な場面が多い。
人間と計算機やセンサ・アクチュエータを処理に応じて使い分けることによって、より効率的であったり、正確な処理が実現する。

プログラムとして記述することで、様々な情報を取得が容易になる。
Aという処理にかかった時間は、少し前のプログラムの場合は1分であったが、改善することで30秒にすることができた、
というようなことは、解析プログラムを導入することですぐにわかるだろう。
AさんとBさんが同じ処理を実行した場合の実行時間の差などもデータとして取得することが容易となる。
改善の結果が悪ければすぐに元のプログラムに戻す、といったことも、バージョン管理が一般的に行なわれている
プログラムだからこそ、実行しやすい。

また、複数の人間をオーケストレーションし、効率的に動かすといったことも可能になる。
マニュアルなどでは、基本的に一人の人間を対象としている。
しかし、プログラムで記述し、各自に指示を送るようにすれば、複数の人間の指揮も容易であると考えられる。

\section{型指定}\label{ux578bux6307ux5b9a}

Babascriptプログラミング環境においては、人間への指示に返り値の型を指定することができる。
型はBoolean、String、Number、Selectの4種類が存在し、ユーザにはその型に合ったインタフェースが提示される。

型ごとに入力コストが異なるため、指示の実行終了までの時間が変化すると考えられる。
例えば、String型を指定した場合、文字を入力しなくてはならない。
入力する文字を考える時間や文字入力のスピード、文字入力システムの性能等にも左右される。
一方、Boolean型を指定した場合、2択から選んで決定するだけである。
String型とBoolean型では入力コストに差が存在し、他の2つの型においても同様のことが言える。
可能な限り早く値を返して欲しい場合などは、Boolean型を指定することが望ましい。

型を指定されることによって、ユーザ側も入力すべき値がはっきりとし、迷うことがなかったという意見があった。
一方で、String型の場合、指示内容にもよるが入力の自由度が高くなりがちという問題がある。
String型においては、どんな値を入力すれば良いか迷ったり、わからなかったという意見もあった。
これは、どんな値を入力すべきかという例示をすることによって、ある程度解決可能な問題だと考えられる。

\section{処理の実行遅延と実行保証}\label{ux51e6ux7406ux306eux5b9fux884cux9045ux5ef6ux3068ux5b9fux884cux4fddux8a3c}

Babascriptを用いて人に指示を送っても、指示を受け取った人がその場ですぐに処理を実行するとは限らず、
遅延して実行される可能性がある。
指示を受け取るデバイスを見ていないという場合や、指示を受け取っても状況的にすぐに実行できないという場合があると
考えられ、その場合はすぐに指示に対して実行結果を返すことが出来ない。
Babascriptでは、こういった状況に対応するため、非同期実行を前提とした設計にしている。
そのため、人間による処理の実行中はコンピュータ側に他の処理を実行させておくといったことも可能だが、
人間による処理はほぼ確実に全体の処理を遅延させると考えられる。

また、指示を無視するといったことが起きる可能性があり、実行を完全に保証することができない。
上司からの命令などのように、労働上のある程度の強制力がある場合や、
自分自身への指示、家族からの指示などの場合は、無視の可能性は低くなるが、
強制力がない場合は、指示を無視するといったことが起こりうる。
こういった状況においては、何かしらのインセンティブをワーカーに与えることによって実行保障性を確保することが
できると考えられる。

\section{エラーへの対応}\label{ux30a8ux30e9ux30fcux3078ux306eux5bfeux5fdc}

Babascriptからの指示を受け取った時、その指示が想定している状況と、現実の状況が大きく乖離している可能性がある。
場所や時間に依存するような指示の場合等に発生することが予想される。
指示を受け取った時点や場所においては、指定された型では適切な値を返せないといったことが考えられる。
また、無理やり値を返そうとした結果、本来返されるべき値をは異なる値が処理結果として入力されてしまう可能性も存在する。

Babascript
Agentには、実行出来なかった際などに利用するエラー入力用のインタフェースが存在する。
エラーが起きた際、この機能を適切に利用することでプログラム側に通知が可能だ。
現状の実装では、エラーの内容を文字列として入力するか、デフォルトで指定可能な3つの選択肢から選ぶしかない。
エラーの内容を文字列として表現することは難しく、実際に起きたこととは異なるエラーが報告されてしまう可能性もある。
また、ワーカーへの負担も大きくなる。

また、エラーでプログラムが終了した場合、途中から実行しなおすといったことが現状では出来ない。
実世界におけるタスクの多くは、途中で間違っても、その間違いがタスク継続が困難なものでない限り、途中からやり直すといったことが可能だ。
Babascriptで実現するプログラムにおいても、途中からやり直すという機構が求められる。
Turkit\cite{turkit}におけるcrash-and-rerun
programmingの概念のように、途中経過を保存しておくことで、
プログラムの途中から実行し直すという仕組みが実現可能であると考える。
また、途中経過保存によって、どんなタスクがエラーを起こしやすいのかといったことも保存することができ、
ワークフロー改善にも役立つと考えられる。

\section{処理の先送り}\label{ux51e6ux7406ux306eux5148ux9001ux308a}

プログラムから指示を受け取った時、今でなくても後でなら処理できるという状況が存在する。
しかし、そのまま指示を放置していては、タイムアウト処理等によって指示が撤回されてしまったり、
他の処理が遅延してしまう可能性が高い。
処理の先送りをプログラムに通知しておき、またあとで指示を通知し直すなどの仕組みを用意する必要がある。

これは、Babascirpt
Agent側にも非同期的に指示を通知するインタフェースが必要となる、ということである。
現状では、Babascript Agentは完全に同期的な挙動をする。
先送りした指示や現時点でキューに溜まっている指示などをリスト型のインタフェースで表示しておき、
実行する処理を選べるような設計をすることで、Babascript
Agent側で処理の先送りを実現可能である。

\section{割り込み処理}\label{ux5272ux308aux8fbcux307fux51e6ux7406}

コンピュータが多くの割り込み処理を行っているのと同じように、
様々な指示が送られる中で、人間に対しても、指示を割り込みできるような仕組みが求められる。
例えば、現実においても、割り込み処理は多くなされている。
料理をしてる時に、鍋が吹きこぼれそうだったら、他の処理を中断してでも鍋から吹きこぼれないような処理を行う。

このような割り込み処理を、Babascriptでは、人間の指示構文実行時のオプション情報として
特別なフラグを立てることで実現させている。
人間への指示は、Node-Linda上にキュー形式で保存されているが、
この指示のキューを通常タスクと優先タスクで分けて保存している。
優先タスクキューに指示が入ってる場合は、こちらのキューから優先して指示を取得する。

割り込み処理に関しては、Babascript Agent 側の工夫が非常に重要になる。
実世界における割り込み処理の通知は非常にわかりやすい。
鍋が吹きこぼれそうなら、視覚的にわかりやすく通知が行なわれ、
機械がエラーで止まれば、エラー音による通知が行われる。 Babascript
Agentにおいても、割り込み処理が来た際にはわかりやすく通知することが望ましい。
現状では、高度な工夫は実現できていない。
割り込み処理であるということを明示的にし、割り込んだ処理をすぐに行うように
誘導させられるようにインタフェースを再実装する必要がある。

\section{命令の抽象度設計の必要性}\label{ux547dux4ee4ux306eux62bdux8c61ux5ea6ux8a2dux8a08ux306eux5fc5ux8981ux6027}

Babascriptでは、人への指示構文の記述には制限がない。
そのため、命令の抽象度はプログラマの指示記述能力に依存する。
抽象度が高すぎる指示内容にしてしまうと、ワーカーにとって実行内容が理解しづらくなってしまう。
結果として、想定外の処理が実行されたり、意図しない処理結果を返される可能性が存在する。
人間は柔軟な解釈が可能なため、ある程度は補正可能だと考えられるが、補正が不可能なほど抽象的な
指示内容の場合、問題が発生する。

具体的過ぎる命令は、全体の処理内容にもよるが、プログラム自体が冗長となり得る。
プログラムとワーカーの間のやりとりが増え、通信や待機時間、入力時間などによって処理全体が遅延すると考えられる。
また、ワーカーにとっても、やりとりが増えることで、入力の手間が増え、負担増に繋がる。

指示ごとに異なると考えられるが、命令文は適切な抽象度に設計しなくてはならない。

\section{異なる種類の指示の複数同時実行}\label{ux7570ux306aux308bux7a2eux985eux306eux6307ux793aux306eux8907ux6570ux540cux6642ux5b9fux884c}

複数のプログラムから同時に一人の人間に複数の関係のない指示が実行される可能性がある。
例えば、料理プログラムと掃除プログラムが同時に実行された場合、
鍋で何かを煮ている途中に、「洗剤を投入する」といった指示がなされるといったことが考えられる。

現実世界で同じようなことが起きても、人間は指示を分類し、別スレッドで動作させることで問題を回避する。
Babascript Agent
Webアプリケーションプロトタイプ2では、指示を並列に表示させ、
人間側が自分で記憶を元に分類しておくことでこの問題を部分的に解決している。
しかし、類似した指示が来た場合、混同してしまう可能性がある。
どのプログラムからの指示なのかを区別し、
分類し提示するインタフェースを実装することで解決可能と考えられる。

\section{Babascript
Agentへのキャラクター性の付与}\label{babascript-agentux3078ux306eux30adux30e3ux30e9ux30afux30bfux30fcux6027ux306eux4ed8ux4e0e}

Babascript Agentは、一種のソフトウェアエージェントである。 Babascript
Agentを用いた、人とプログラムのコミュニケーションは
ヒューマンエージェントインタラクションであると言える。
ヒューマンエージェントインタラクションとは、
人間とソフトウェアエージェントやロボット間のインタラクション
設計に関する研究分野である。
ヒューマンエージェントインタラクション研究の分野においては、ソフトウェアエージェントやロボットに対して
擬人化手法を適応させることによって、エージェントをより人間らしく見せたり、コミュニケーションを取りやすく
するといった研究がなされている。

擬人化エージェントがもたらす効果については、村上らによる研究\cite{murakami}では、エージェントに人間関係を適応させた実験を行っている。
村上らの実験では、研究室の教官と学生の関係を利用し、教授の顔を模したキャラクターと関係のないキャラクター、両者からの
依頼に対して学生の行動が変化するかどうかを観察した。
実験の結果、教官キャラクターのほうがより依頼を受理されやすいという結果がわかった。
この実験結果から、エージェントに人間関係、特に上司などを模したキャラクターをユーザに提示することである程度、
依頼に対してポジティブにとらえてもらうことが可能だと考えられる。

上記の実験結果は、Babascript Agentにも応用可能である。 Babascript
Agentにキャラクター性を持たせ、そのキャラクターとコミュニケーションを取っているかのように見せる
インタフェースを開発すれば、指示が受理されやすくなる可能性が存在する。
また、キャラクター性の付与によって、プログラムからの指示であるという印象が薄れ、
指示に対する拒否感も薄れるといったことが考えられる。

\section{今後の展望}\label{ux4ecaux5f8cux306eux5c55ux671b}

今後は、本章で述べた問題点についての改善を主に行う。
割り込み処理や処理の先送りは特にBabascript Agent側の改良が必要である。
エラー処理に関する問題も、非常に興味深く、利便性を上げるために解決が必須である。
また、Babascript
Agentへのキャラクター性付与による人間の心理面の変化など、実験の上、その効果を検証したい。
