\chapter{議論}\label{chap:discussion}

\section{プログラム化する意味}\label{ux30d7ux30edux30b0ux30e9ux30e0ux5316ux3059ux308bux610fux5473}

超重要

\begin{itemize}
\itemsep1pt\parskip0pt\parsep0pt
\item
  他の様々なプログラムと連携できる
\item
  人間のデータが取れる
\item
  バージョン管理が容易
\item
  ロボットとかに処理が置き換えられてもすぐに対応可能
\end{itemize}

\section{処理への人の積極的な貢献}\label{ux51e6ux7406ux3078ux306eux4ebaux306eux7a4dux6975ux7684ux306aux8ca2ux732e}

\section{処理単位としての人}\label{ux51e6ux7406ux5358ux4f4dux3068ux3057ux3066ux306eux4eba}

本研究では、人は処理を命令され実行する存在となり、プログラム上においてコンピュータやセンサーと同等となる。
こういったことに対して心理的な拒否感を覚えることも考えられる。
しかし、一処理単位として扱うことは、大きなメリットでもある。
命令を実行するだけのノードであるということは、ただ処理内容を実行することにのみ集中すれば良いということだ。
ただやるべきことだけが提示され、その通りに動けば良いということは、深く考える必要がなく、楽であるということが考えられる。
もちろん、全ての処理がただ提示する通りに動けば良いものではないと考えられるが、
作業を楽にするための一つのアイデアであると言える。

\section{プログラムと身近な人の違い}\label{ux30d7ux30edux30b0ux30e9ux30e0ux3068ux8eabux8fd1ux306aux4ebaux306eux9055ux3044}

プログラムからの指示でも、身近な人からの指示でも、たいして変わらないのでは
指示元を隠蔽するような仕組みがあれば、違和感なくできるのでは?

\section{フォーマットによる違い}\label{ux30d5ux30a9ux30fcux30deux30c3ux30c8ux306bux3088ux308bux9055ux3044}

ユーザ評価実験の結果、返り値フォーマットの違いがある程度

\section{処理の実行遅延と実行保証}\label{ux51e6ux7406ux306eux5b9fux884cux9045ux5ef6ux3068ux5b9fux884cux4fddux8a3c}

Babascriptによる人への命令構文が実行されても、命令を受け取った人がすぐに命令に対する処理を実行し値を返すことを完全に保証することはできない。
タスク受信端末を見ていない、受信しても実行できないといった状況の場合、すぐに値を返すことはできない。
Babascriptによる人への命令構文は全て非同期実行する実装となっているため、人力処理の待ち時間に他の処理を実行させておくことが可能であるが、人力処理の実行待ちが全体の処理を遅延させる可能性は存在する。

また、労働関係にあるなど、強制力がある場合命令は実行されると考えられるが、強制力がない場合はそもそもタスクを無視するといったことも考えられる。
タスク実行に強制力がない場合は、金銭などのインセンティブを与えるといった手段によって、実行保障性を確保するといったことが考えられる。

\section{例外処理機構}\label{ux4f8bux5916ux51e6ux7406ux6a5fux69cb}

Babascriptにおいて、命令が想定する状況と現実の状況との乖離によって適切な返り値を選択・記述ができなくなるといった可能性がある。
これは、現実が刻々と変化していることなどから、完全に避ける事の出来ない問題であると考える。
この際、無理やり値を返すといった処理をしてしまうと、本来の状態とは違った判断がなされてしまう危険性がある。
命令文とは明らかに現実が異なっている場合などは、タスク実行者から例外としてプログラムに通知出来るような仕組みの実装によって、問題の解決へと繋げられると考える。

\section{命令の先送り}\label{ux547dux4ee4ux306eux5148ux9001ux308a}

人間とコンピュータの違いに、タスクの管理手法が異なるという点がある。
コンピュータはプログラムの通りに処理を管理し、実行していくが、人間は違う。
気分であったり、状況や環境に応じてタスクの処理手順を前後させるといったことを行っている。
こういった人間の特性についても、仕組みとして取り入れる必要がある。

人間は非同期的存在である。
そのため、各種命令をあとで処理する、といったことができないといけない。

\section{命令の抽象度設計の必要性}\label{ux547dux4ee4ux306eux62bdux8c61ux5ea6ux8a2dux8a08ux306eux5fc5ux8981ux6027}

Babascriptでは、人への命令構文の記述には制限がない。
そのため、命令の抽象度はプログラマに依存する。
抽象度が高すぎるメソッド名にしてしまうと、タスク実行者にとって理解しづらい文面となり得る。
その結果、想定外の処理が実行され、意図しない結果を招く恐れがある。

具体的過ぎる命令は、全体の処理内容にもよるが、プログラム自体が冗長となり得る。
プログラムとタスク実行者の間のやりとりが増え、通信や待機時間などがボトルネックとなる可能性がある。
また、タスク実行者にとっても、やりとりが増えることで負担増になると考えられる。
処理ごとに異なると考えられるが、命令文は適切な抽象度に設計しなくてはならない。

\section{複数命令の同時実行}\label{ux8907ux6570ux547dux4ee4ux306eux540cux6642ux5b9fux884c}

複数のプログラムから同時に一人のタスク実行者へとタスクが配信される可能性がある。
この際、異なるコンテキストにある命令が交互に配信され、タスク実行に大きな障害をもたらす可能性がある。
例えば、料理プログラムと掃除プログラムが同時に実行された場合、鍋で煮ている途中に「洗剤を投入しろ」などといった命令が配信されることが考えられる。

この問題は、命令実行者は一つのプログラムからのみ連続して命令を受信できるような仕組みを用意することによって、解決可能であると考えられる。
また、応用アプリケーションでの実装になるが、コンテキストを明示し、どの処理系におけるタスクなのかを命令実行者に示すといった手段によっても解決可能である。

\section{クライアントアプリケーションへのキャラクター性の付与}\label{ux30afux30e9ux30a4ux30a2ux30f3ux30c8ux30a2ux30d7ux30eaux30b1ux30fcux30b7ux30e7ux30f3ux3078ux306eux30adux30e3ux30e9ux30afux30bfux30fcux6027ux306eux4ed8ux4e0e}

\section{まとめ}\label{ux307eux3068ux3081}
