\chapter{応用例}\label{chap:application}

本章では、Babascriptプログラミング環境によって実現可能と考えられる応用例について述べる

\section{仕事のプログラム化}\label{ux4ed5ux4e8bux306eux30d7ux30edux30b0ux30e9ux30e0ux5316}

人の仕事は、マニュアルのような形で記述されることが多い。
マニュアルはどのように行動すべきかが構造的に記述された手順書であり、「Aという条件の時にはBの処理を実行する」といったことが文章として記述されていることが多い。
コンピュータにとっての手順書であるプログラムとは類似点多く、プログラムに変換可能であると考える。

プログラム化することで、仕事における進行度などの状態をプログラムで管理することができる。
状態を可能な限りプログラムで管理し、経験や知識によって可能な条件判断などもプログラム化することができれば、プログラムに指示されたことのみを実行するだけである程度の仕事を実行できると考える。
指示されたことのみを実行するだけでも良いということは、仕事の引き継ぎなどが最低限となり、人の代替をも容易にする可能性がある。
プログラム経由にすることで、詳細な実行ログや実行状況を把握することも可能である。
仕事の実行量の定量化や、状況監視、状況の可視化などに応用可能であると考える。

また、人同士がコミュニケーションを取ることなく、複数人を協調させるといったことも可能となる。
通常、複数人を協調させるためには、人同士が相談したり、上位の意思決定者が必要となる。
しかし、コミュニケーションはコストのかかるものであり、適切に行われない場合、問題が生じることもある。
意思決定をプログラムに委ねることは、複数人を効率よく協調させることに繋がるとも考えられる。

仕事の定量的判断につながる

\section{人の力を利用した実世界プログラミング}\label{ux4ebaux306eux529bux3092ux5229ux7528ux3057ux305fux5b9fux4e16ux754cux30d7ux30edux30b0ux30e9ux30dfux30f3ux30b0}

人をセンサーやアクチュエータとして利用することで、人を使って実世界を操作することが可能となる。
現在のセンサーやアクチュエータでは、実世界に干渉するのには限界がある。
人力処理を組み込むことのできる本提案手法ならば、例えば、コンテキスト情報を伴うようなセンシングであっても、実現可能である。
また、人を汎用的に利用可能なアクチュエータとして利用することもできる。
人をアクチュエータとして利用する、Human as Actuator
と言えるような事例は少ない。
両者を組み合わせることによって、既存の仕組みでは困難であった実世界の操作も可能となると考えられる。

\section{実世界環境の構成管理とテスト}\label{ux5b9fux4e16ux754cux74b0ux5883ux306eux69cbux6210ux7ba1ux7406ux3068ux30c6ux30b9ux30c8}

実世界環境は今後プログラムと密結合していく。
ユビキタスコンピューティング等の実現によって、実世界環境にコンピュータが多く存在し、
プログラムによって制御される空間が実現する。
空間とプログラムは密に結合し、両者を分けて考える事は難しくなると考えられる。
このような実世界環境においても、構成管理やテストをする仕組みが求められる。
つまり、その空間を構成するプログラムとモノであれば、同じようにコードとして記述して構成管理をする必要がある。
しかし、従来の枠組みでは、プログラムの部分のみの構成管理やテストしかできない。
そこでBabascript環境を使うことで、実世界とプログラムの双方を対象とすることができる。
プログラム部分はコンピュータに、実世界の部分に関しては人間によって構成管理を実行したり、テストを行う。

この考えは、サーバやインフラの構成をコードとして記述し、管理するInfrastructure
as Codeという概念を基礎にしている。
従来では、サーバのセットアップは手順書などを元に手動で行っていたが、これをコードとして記述し実行させるというものである。
サーバやインフラの構成をコードで記述しておけば、様々なイベントを元に自動実行が可能であったり、バージョン管理が容易になるため、
新しい構成の構築に失敗しても、すぐに前のバージョンに戻してセットアップが行える。
サーバの構成をテストする仕組みと組み合わせることによって、サーバやインフラさえも、継続的インテグレーションに組み込むことができる。
このように、手順をコードで記述することによって、自動化やテストが可能になるなど有用であることが多い。

空間を制御するプログラムと、そのプログラムが実行されるコンピュータや制御する対象等を含んだ実世界の構成管理と
テストが実現することによって、様々なメリットをユーザは享受出来ると考えられる。
空間の構成がプログラム化されることによって、その空間においてどのような仕組みが動いているのかを形式知として残すことができる。
例えば、増井研究室があるデルタS112という空間では、様々なプログラムが動いており、そのプログラムが動くコンピュータや
制御する対象も様々であるが、その情報は暗黙知に近い状態であると言える。
もし、引っ越しをしたりする場合、実行されなくなってしまうプログラム等も存在すると考えられる。
コンピュータを買い替えたりした場合も同様である。
これらの情報を形式知化し、バージョン管理していくことが出来れば、非常に有用であると言える。

\section{まとめ}\label{ux307eux3068ux3081}

本章では、本提案によって実現が可能と考える応用例について述べた。
人間を利用できることで、今までは実現が困難であったような処理であっても、実現は可能である。
