\chapter{応用例}\label{chap:application}

本章では、Babascriptプログラミング環境によって実現可能と考えられる応用例について述べる

\section{Activity as Code}\label{activity-as-code}

Babascriptプログラミング環境によって、プログラムはコンピュータへの指示系統だけでなく、人間への指示系統を利用できるようになる。

\section{仕事のプログラム化}\label{ux4ed5ux4e8bux306eux30d7ux30edux30b0ux30e9ux30e0ux5316}

人への指示がプログラムとして記述可能になることで、仕事をプログラム化することができると考える。
人の仕事は、例えば、マニュアルのような形で記述されることが多い。
マニュアルはどのように行動すべきかが構造的に記述された手順書であり、「Aという条件の時にはBの処理を実行する」といったことが文章として記述されていることが多い。
コンピュータにとっての手順書であるプログラムとは類似点多く、プログラムに変換可能であると考える。

プログラム化することで、仕事における進行度などの状態をプログラムで管理することができる。
状態を可能な限りプログラムで管理し、経験や知識によって可能な条件判断などもプログラム化することができれば、プログラムに指示されたことのみを実行するだけである程度の仕事を実行できると考える。
指示されたことのみを実行するだけでも良いということは、仕事の引き継ぎなどが最低限となり、人の代替をも容易にする可能性がある。
プログラム経由にすることで、詳細な実行ログや実行状況を把握することも可能である。
仕事の実行量の定量化や、状況監視、状況の可視化などに応用可能であると考える。

また、人同士がコミュニケーションを取ることなく、複数人を協調させるといったことも可能となる。
通常、複数人を協調させるためには、人同士が相談したり、上位の意思決定者が必要となる。
しかし、コミュニケーションはコストのかかるものであり、適切に行われない場合、問題が生じることもある。
意思決定をプログラムに委ねることは、複数人を効率よく協調させることに繋がるとも考えられる。

\section{人の力を利用した実世界プログラミング}\label{ux4ebaux306eux529bux3092ux5229ux7528ux3057ux305fux5b9fux4e16ux754cux30d7ux30edux30b0ux30e9ux30dfux30f3ux30b0}

人をセンサーやアクチュエータとして利用することで、人を使って実世界を操作することが可能となる。
現在のセンサーやアクチュエータでは、実世界に干渉するのには限界がある。
人力処理を組み込むことのできる本提案手法ならば、例えば、コンテキスト情報を伴うようなセンシングであっても、実現可能である。
また、人を汎用的に利用可能なアクチュエータとして利用することもできる。
人をアクチュエータとして利用する、Human as Actuator
と言えるような事例は少ない。
両者を組み合わせることによって、既存の仕組みでは困難であった実世界の操作も可能となると考えられる。

\section{実世界環境の構成管理とテスト}\label{ux5b9fux4e16ux754cux74b0ux5883ux306eux69cbux6210ux7ba1ux7406ux3068ux30c6ux30b9ux30c8}

近年、Infrastructure as Code
と言った言葉があるように、サーバの構成をコードとして管理するといったことが一般化している。
また、サーバの構成が適切であるかを確認するために、サーバ自体をテストするといった仕組みも一般化している。
コードとして記述しておき実行することで、可能な限り人的なミスを減らすことができる。
また、バージョン管理やソーシャルコーディングが容易になるなど、サーバの構成をコードとして記述し管理・テストする手法は有用である。

Babascriptプログラミング環境を利用することによって、この考えを、実世界環境に適応することが可能となる。

\section{まとめ}\label{ux307eux3068ux3081}
