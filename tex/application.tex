\chapter{応用例}\label{chap:application}

本章では、Babascriptプログラミング環境によって実現可能と考えられる応用例について述べる。

\newpage

\section{仕事のプログラム化}\label{ux4ed5ux4e8bux306eux30d7ux30edux30b0ux30e9ux30e0ux5316}

第\ref{chap:background}章でも述べた通り、人間の仕事はマニュアル等によって記述されていることが多い。
マニュアルはどのように行動すべきかが構造的に記述された手順書であり、
「Aという条件の時にはBの処理を実行する」といったことが文章として記述されている。
そのため、これらの手順をソースコードに変換し、プログラムとして実行することは可能と考える。

プログラムとして実行することで、仕事における様々な「状態」を計算機に管理させることができる。
この「状態」とは、例えば、進行度が挙げられる。
今までは働いている人間の記憶や知識に依存していた部分を計算機に管理させることで、
可能な限りの条件判断を人間が行う処理から外すことが出来る。
計算機の判断の元、人間に行動指示が出され、仕事を進行させていくことが可能である。
これにより、個人の経験や知識への依存を減らすことが出来る。
個人への依存は人の代替や引き継ぎを困難にするため、計算機による支援を積極的に行うことで
回避すべきである。

また、プログラム経由での情報のやりとりが増えるため、詳細な実行ログや実行状況をデータとして取得することができる。
これはつまり、仕事の実行量の計測が可能ということである。
作業者がどのように働いていて、どのくらい時間がかかっているのかという情報は、作業者の行動改善に役立つ。
データを元に状況の可視化なども可能である。

プログラムを介することで、人間同士がコミュニケーションを取ることなく、協調して仕事を進めていくことが出来る。
通常、仕事で複数の人が関係してくる場合、当事者同士での相談等、コミュニケーションコストがかかってしまう。
適切に行なわれない場合、遅延等の問題が発生する。
意思決定を計算機に委ね、人間による意思決定が必要な場面においては人間に意思判断をさせることで、
コミュニケーションコストを最低限にしつつ複数人による仕事の実行が可能となると考えられる。

\section{人の力を利用した実世界プログラミング}\label{ux4ebaux306eux529bux3092ux5229ux7528ux3057ux305fux5b9fux4e16ux754cux30d7ux30edux30b0ux30e9ux30dfux30f3ux30b0}

人間をセンサーやアクチュエータとして利用することで、人間を使って実世界から情報を得たり、操作することができる。
人間は優秀な入出力装置であり、計算機だけでは現在は実現が困難な処理でも実行できるケースがある。
例えば、現在のセンサー技術では、その場の雰囲気の数値化や文字列化といったコンテキストを伴うようなセンシングは困難である。
人間であれば、コンテキストを伴うようなセンシングでも柔軟に解釈し、情報を得ることが出来る。
また、現在のアクチュエータでは、簡単な動作を実行することは容易であるが、複雑な動作となると決められた動きを実行する等限定される。
人間であれば、多少あいまいな指示でも複雑な動作を実現できる。

もちろん、人間が不得意なことやコストを十分に考慮する必要がある。
例えば、正確な温度等の情報を取得するようなことは機械のセンサーを利用するべきである。
人間は知能によって処理が必要となるような情報取得のほうが得意である。
また、定期的に何かを動かしたりすることは、機械のアクチュエータがするべきである。
繰り返し処理や単純な処理は計算機のほうが得意である。

人間とセンサーやアクチュエータを状況に応じて使い分けるという手法も考えられる。
例えば、部屋Aにはセンサーが存在するが、部屋Bにはセンサーが存在しないとして、
部屋Aでプログラムが実行された場合は機械のセンサーを、部屋Bで実行された場合は人間によって
情報の取得を行うというプログラムが実現可能である。

\section{実世界環境の構成管理とテスト}\label{ux5b9fux4e16ux754cux74b0ux5883ux306eux69cbux6210ux7ba1ux7406ux3068ux30c6ux30b9ux30c8}

実世界環境は今後プログラムと密結合していく。
ユビキタスコンピューティング等の実現によって、実世界環境に計算機が多く存在し、
プログラムによって制御される空間が実現する。
空間とプログラムは密に結合し、両者を分けて考える事は難しくなると考えられる。
このような実世界環境においても、構成管理やテストをする仕組みが求められる。
つまり、その空間を構成するプログラムとモノであれば、同じようにコードとして記述して構成管理をする必要がある。
しかし、従来の枠組みでは、プログラムの部分のみの構成管理やテストしかできない。
そこでBabascript環境を使うことで、実世界とプログラムの双方を対象とすることができる。
プログラム部分は計算機に、実世界の部分に関しては人間によって構成管理を実行したり、テストを行う。

この考えは、サーバやインフラの構成をコードとして記述し、管理するInfrastructure
as Codeという概念を基礎にしている。
従来では、サーバのセットアップは手順書などを元に手動で行っていたが、これをコードとして記述し実行させるというものである。
サーバやインフラの構成をコードで記述しておけば、様々なイベントを元に自動実行が可能であったり、バージョン管理が容易になるため、
新しい構成の構築に失敗しても、すぐに前のバージョンに戻してセットアップが行える。
サーバの構成をテストする仕組みと組み合わせることによって、サーバやインフラさえも、継続的インテグレーションに組み込むことができる。
このように、手順をコードで記述することによって、自動化やテストが可能になるなど有用であることが多い。

空間を制御するプログラムと、そのプログラムが実行される計算機や制御する対象等を含んだ実世界の構成管理と
テストが実現することによって、様々なメリットをユーザは享受出来ると考えられる。
空間の構成がプログラム化されることによって、その空間においてどのような仕組みが動いているのかを形式知として残すことができる。
例えば、増井研究室があるデルタS112という空間では、様々なプログラムが動いており、そのプログラムが動く計算機や
制御する対象も様々であるが、その情報は暗黙知に近い状態であると言える。
もし、引っ越しをしたりする場合、実行されなくなってしまうプログラム等も存在すると考えられる。
これらの情報を形式知化し、バージョン管理していくことが出来れば、非常に有用であると言える。

\section{まとめ}\label{ux307eux3068ux3081}

本章では、本提案によって実現が可能と考える応用例について述べた。
人間を利用できることで、今までは実現が困難であったような処理であっても、実現は可能である。
