\begin{acknowledgment}

修士課程の2年間、研究をご指導いただいてきた慶應義塾大学環境情報学部の増井俊之教授に深く感謝致します。
また、本研究の副査としてご意見、ご助言を頂きました徳田教授、田中准教授に感謝致します。

インタラクションデザインプロジェクトに在籍し、メンバーたちと多くの議論をすることができました。
橋本翔さんには、初期の頃よりBabascriptについて多くの議論していただきました。
橋本さんがいなければ、Babascriptが研究として発表されることはなかったと思われます。感謝いたします。
研究・私生活ともに、同期の皆様には様々な面において支えられてきました。
臼杵壮也さん、郡山隼人さん,田中優さんに感謝します。
中園翔さんと桜井雄介さんには、iDPのメンバーとして様々な意見を頂きました。
また、中園さんには本論文の校正も手伝っていただきました。
感謝します。
増井研究室の学部生の皆様にも感謝します。

本研究は、独立行政法人情報処理推進機構 2013年度未踏IT人材発掘・育成事業の支援を受けました。
未踏事業の中でプロジェクトマネージャーとして多くの助言を頂きました大阪大学の石黒浩教授に感謝致します。
未踏事業の同期のメンバーとは、研究を進めていく上で重要となった議論を多く交わすことができました。
また、多くのOBの方々にもアドバイスを頂きました。感謝致します。

修士論文執筆にあたっては、学部時代の先輩にあたる黒井さんと山本伶さんが作ったテンプレートを使用させていただきました。
また、山本伶さんには、本論文の校正を手伝っていただきました。感謝いたします。
片倉弘貴さんには、在学中にプログラミングに関して、Babascriptに関して多くのアドバイスを頂きました。
片倉さんとの議論は本研究に大きな影響を与えています。また、本論文の校正も手伝ってもらいました。感謝します。
秋山博紀さんには、学部生の頃より様々なアドバイスを頂きました。
また、元イオタの民たちには、修士生活で悩んだことの多くを相談させていただき、
的確なアドバイスをいただきました。感謝いたします。

最後に、大学に泊まることが多く、中々家に帰ってこない私の身をいつも心配してくれていた両親と、
実装に関するアドバイスを始め、技術的な話に付き合ってくれた兄に感謝いたします。ありがとうございます。

\end{acknowledgment}
