\begin{jabstract}

% 本研究では、人間と計算機の処理を融合させることを目的としたプログラミング環境を提案する。
% 近年、人間を計算資源としてシステムに組み込み利用するヒューマンコンピュテーションの研究が流行している。
% 人間にしかできないことは多く存在しており、人間と計算機が協働し問題解決を行う
% しかし、既存の研究では不特定の人間の知能を利用することを主眼としており、
% 
% そこで、本研究では、人間にしかできないことを人間に実行させるために、

%
% 具体的には、人間への指示を記述可能にするプログラミングライブラリであるBabascript、
% 人間とプログラムのインタフェースとして機能するBabascript Agent、
% 仲介サーバとして機能するNode-Lindaから構成されるBabascriptプログラミング環境を提案する。
% このプログラミング環境によって、

プログラム上で人間と計算機への指示を同等に記述可能なシステムについて述べる。
人間を計算資源としてシステムに組み込み利用するヒューマンコンピュテーションの研究が流行している。
しかし、その多くは不特定多数の人間を演算装置として利用するものであり、入出力装置として利用していない。
例えば自分自身など、具体的な人物を指定することができず、実世界におけるタスク処理には向いていない。

このような状況を踏まえ、本研究では、センシングやアクチュエーションを含んだ人間への行動指示と指示結果の取得を
従来のプログラミング言語に組み込んだシステムを提案する。
このシステムにより、シンプルな記述法で人間と計算機への指示が混在したプログラムを記述・実行することができるようになり、
計算機が得意なことは計算機が、人間が得意なことは人間が実行するという、人間と計算機の協働を実現する。

\end{jabstract}

\begin{eabstract}

In this research, We introduce a programming environment that supports the integration of human resources and compuing resources.
Although HumanComputation system is becoming popular,
I can't write english well...
ganbatte kakukaku

\end{eabstract}
