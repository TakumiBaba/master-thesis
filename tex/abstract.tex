\begin{jabstract}

プログラム上で人間と計算機への指示を同じように記述可能なシステムについて述べる。
人間を計算資源としてシステムに組み込み利用するヒューマンコンピュテーションの研究が流行している。
しかし、その多くは人間を演算装置として利用するものであり、入出力装置として利用していない。
また、例えば自分自身など、具体的な人を指定することができず、実世界におけるタスクの処理などには向いていない。

このような状況を踏まえ、本研究では、センシングやアクチュエーションを含んだ人間への行動指示と指示結果の取得の仕組みを
従来のプログラミング言語に組み込んだシステムを提案する。
よりシンプルな記述法で、人間と計算機への指示が混ざったプログラムを記述・実行することができる。
このシステムによって、世の中に存在する様々な処理を
計算機が得意なことは計算機が、人間が得意なことは人間が実行するという、人間と計算機の共生を実現する。
<!-- 人間と計算機の共生という言葉をやめたい。 -->

\end{jabstract}

\begin{eabstract}

In this research, We introduce a programming environment that supports the integration of human resources and compuing resources.
Although HumanComputation system is becoming popular,
I can't write english well...
ganbatte kakukaku

\end{eabstract}
