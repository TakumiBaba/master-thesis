\chapter{序論}\label{chap:introduction}

\section{研究の目的と概要}\label{ux7814ux7a76ux306eux76eeux7684ux3068ux6982ux8981}

プログラムは様々な制御を行うようになっている
プログラムは様々な処理を記述できる
プログラムは汎用的処理記述フォーマットとすることもできる

本論文では、

\section{用語定義}\label{ux7528ux8a9eux5b9aux7fa9}

本論文において使用する用語を以下のように定義する。

\begin{itemize}
\itemsep1pt\parskip0pt\parsep0pt
\item
  ヒューマンコンピュテーション
\item
  クラウドソーシング
\item
  アウトソーシング
\end{itemize}

\paragraph{計算資源}\label{ux8a08ux7b97ux8cc7ux6e90}

計算資源とは、計算機が情報処理のために利用する機器を示す。

\section{本論文の構成}\label{ux672cux8ad6ux6587ux306eux69cbux6210}

第\ref{chap:background}章では、背景となるプログラムの意味について整理し、プログラムが記述する領域について述べる。

第\ref{chap:design}章では、人と計算機を同様に記述可能なプログラミング環境の設計について述べる。

第\ref{chap:implementation}章では、。

第\ref{chap:evaluation}章では、システム評価について述べる。

第\ref{chap:application}章では、本提案で実現するプログラミング環境を利用して実現可能と考えられる応用例について述べる。

第\ref{chap:discussion}章では、本提案に関する考察について述べる。

第\ref{chap:related}章では、関連する研究分野についてまとめる。

第\ref{chap:conclusion}章。
