\chapter{序論}\label{chap:introduction}

\section{研究の動機と目的}\label{ux7814ux7a76ux306eux52d5ux6a5fux3068ux76eeux7684}

プログラムは非常に優れた処理記述フォーマットである。
コンピュータに対する処理命令を記述するためのものであるため、コンピュータが理解できるような、正確な記述が必要である。
一方で、プログラムは人間が書き、読むこともあるため、人間にとっての可読性も考えた上で記述しておかなければならない。
プログラムはコンピュータが実行しているだけで、本来は、実現したい状態への道筋を描いたドキュメントなのである。
プログラムという優れた媒体であらゆる処理を記述出来ればと考える。

近年では、ヒューマンコンピュテーションの概念が広まり、プログラムの指示の下、人間もコンピュータも同様に指示を受け取り作業を
こなすようになっている。
プログラム上において、人間とコンピュータの垣根は今後取り払われていくと考えられる。
人間とコンピュータが共生していく中で、プログラムによって様々な処理を実行していく社会が実現する。

しかし、現状のシステムでは、プログラムから利用出来る人間の機能は限られている。
システムの多くが、インターネットを介した不特定の人間を対象としているため、基本的に演算能力しか使うことが出来ない。
プログラムによって、特定の個人、例えば自分自身を完全に活用することができれば、今まではプログラムとして記述すると考えていなかったような
領域までもがプログラムで記述できるようになるだろう。

そこで、本研究では、人間と計算機への指示を対等に記述可能なプログラミング環境の実現を目的とする。
このプログラミング環境では、特定の人物を対象とすることができる。
そのため、実世界におけるタスクなどをプログラムで記述し、人間に実行させるということが可能となる。
本研究を通して、人間と計算機を計算資源とした新しいプログラムの可能性を模索する。

\paragraph{計算資源}\label{ux8a08ux7b97ux8cc7ux6e90}

計算資源とは、プログラムが実行中に利用可能な機器類を示す。
本研究においては、人間も計算資源として扱う。

\paragraph{ワーカー}\label{ux30efux30fcux30abux30fc}

プログラムからの指示内容を実行する人間を示す。

\paragraph{ソフトウェアエージェント}\label{ux30bdux30d5ux30c8ux30a6ux30a7ux30a2ux30a8ux30fcux30b8ux30a7ux30f3ux30c8}

ユーザとソフトウェアの

\section{本論文の構成}\label{ux672cux8ad6ux6587ux306eux69cbux6210}

第\ref{chap:background}章では、プログラムの実行対象の広がりや、プログラムが処理する領域の変化に
焦点を当て、これらの変化によって実現が可能となった、人と計算機への指示を融合させたプログラミング環境について述べる。

第\ref{chap:design}章では、人と計算機への指示融合させたプログラミング環境に求められる要件についてまとめる。
また、導き出した要件を元に、新しいプログラミング環境に求められるシステムについて考察する。

第\ref{chap:implementation}章では、第\ref{chap:design}章で検討した新しいプログラミング環境の具体例として
Babascriptプログラミング環境を提案し、このプログラミング環境を構成する各要素の実装について述べる。

第\ref{chap:evaluation}章では、システム評価について述べる。

第\ref{chap:application}章では、人間と計算機への指示を融合させた新しいプログラミング環境によって実現が可能と考えられる
新しい応用例について考察する。

第\ref{chap:discussion}章では、本研究で提案するシステムにおける問題点や改善点等について述べる。

第\ref{chap:related}章では、関連する研究を分野ごとにまとめ、説明を行う。

第\ref{chap:conclusion}章では、本研究の成果をまとめ、今後の展開について述べる。
