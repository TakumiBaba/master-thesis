\chapter{序論}\label{chap:introduction}

\section{研究の目的}\label{ux7814ux7a76ux306eux76eeux7684}

本研究の目的は、プログラムにおいて人間と計算機への指示を同等に記述・実行可能にするプログラミング環境の実現である。

プログラミング言語は非常に優れた処理記述フォーマットである。
プログラミング言語は計算機に実行させたいプログラムを記述するためのものであった。
そのため、計算機が理解できるよう正確に記述する必要がある。
一方で、プログラミング言語は人間も読めるように記述することが良しとされている。
つまり、プログラミング言語は計算機・人間問わず読むことの出来る処理の手順書であり、
実現したい状態への道筋を記述したドキュメントであると言える。

近年ではヒューマンコンピュテーションの概念が広まり、プログラムの指示の下、
人間もコンピュータも同様に指示を受け取り、処理を実行するようになっている。
プログラムに記述された処理を実行する対象は計算機だけでなく、人間にも広がっている。
人間を積極的にプログラムに組み込んでいくような事例が増えるにつれ、
プログラムに記述されている処理を実現するために、人間と計算機を区別して利用することがなくなっていくと考えられる。
プログラムに書かれている処理を実現する上で、人間と計算機が協調して実現していくような
社会が実現する。

しかし、現状のシステムでは、プログラムから利用出来る人間の機能は限られている。
システムの多くが、インターネットを介した不特定の人間を対象としているため、
基本的に誰でもできるような単純な作業にしか使うことができない。
もし、特定の人間、例えば自分自身を対象としてプログラムから活用することができれば、
今まではヒューマンコンピュテーションが対象としていなかったような領域までもをプログラミングできるようになる。
プログラムとして記述し実行することで、計算機と協調しながら様々な処理を
実行できるようになるのではないかと考えた。

\section{人間と計算機を融合するプログラミング環境}\label{ux4ebaux9593ux3068ux8a08ux7b97ux6a5fux3092ux878dux5408ux3059ux308bux30d7ux30edux30b0ux30e9ux30dfux30f3ux30b0ux74b0ux5883}

本研究では、計算機への指示と同じような記述方法で人間への指示を実現させるプログラミング環境を提案する。
このプログラミング環境では、特定の個人を指定してプログラムに組み込み、指示を送ることが出来る。
具体的には、プログラム上で人間とのインタラクションを可能にするプログラムモジュールであるBabascriptと、
プログラムと人間の仲介となるソフトウェアエージェントであるBabascript
Agentを組み合わせる。
分散処理プラットフォームであるNode-Lindaによって指示の配信を実現する。
これらの要素を総合して、Babascriptプログラミング環境と呼ぶ。

本論文では、Babascriptプログラミング環境の設計思想や具体的な実装について述べ、
どのような応用領域が存在するかを示し、有用な点や問題点について述べる。

\section{用語定義}\label{ux7528ux8a9eux5b9aux7fa9}

本論文において使用する用語を以下のように定義する。

\paragraph{計算機}\label{ux8a08ux7b97ux6a5f}

デジタルコンピュータを示す。

\paragraph{計算資源}\label{ux8a08ux7b97ux8cc7ux6e90}

計算資源とは、プログラムが実行中に利用可能な機器類を示す。
入出力装置などもその範疇に入る。
センサーやアクチュエータも計算資源である。
本研究においては、人間も計算資源として扱う。

\paragraph{ワーカー}\label{ux30efux30fcux30abux30fc}

プログラムからの指示内容を実行する人間を示す。

\paragraph{ヒューマンコンピュテーション}\label{ux30d2ux30e5ux30fcux30deux30f3ux30b3ux30f3ux30d4ux30e5ux30c6ux30fcux30b7ux30e7ux30f3}

人間を計算資源としてシステムに組み込み利用する考え方。
主にコンピュータでは処理が困難な問題に対して、
人間の柔軟な思考能力を用いることで解決することに利用される。

\paragraph{クラウドソーシング}\label{ux30afux30e9ux30a6ux30c9ux30bdux30fcux30b7ux30f3ux30b0}

インターネットを介した不特定多数の人間(crowd)を対象に、仕事をアウトソーシングすることを示す。
crowdにアウトソーシングすることから、クラウドソーシングと呼ぶ。

\section{本論文の構成}\label{ux672cux8ad6ux6587ux306eux69cbux6210}

第\ref{chap:background}章では、プログラムの実行対象の広がりや、プログラムが処理する領域の変化に
焦点を当て、これらの変化によって実現が可能となった、人と計算機への指示を融合させたプログラミング環境について述べる。

第\ref{chap:design}章では、提案する新しいプログラミング環境に求められる要件についてまとめる。
また、導き出した要件を元に、新しいプログラミング環境に求められるシステムについて考察する。

第\ref{chap:implementation}章では、第\ref{chap:design}章で検討した新しいプログラミング環境の具体例として
Babascriptプログラミング環境を提案し、このプログラミング環境を構成する各要素の実装について述べる。

第\ref{chap:evaluation}章では、システム評価について述べる。

第\ref{chap:application}章では、人間と計算機への指示を融合させた新しいプログラミング環境によって実現が可能と考えられる
新しい応用例について考察する。

第\ref{chap:discussion}章では、本研究で提案するシステムにおける問題点や改善点等について考察する。

第\ref{chap:related}章では、関連する研究を分野ごとにまとめ、説明を行う。

第\ref{chap:conclusion}章では、本研究の成果をまとめ、今後の展開について述べる。
