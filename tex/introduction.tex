\chapter{序論}\label{chap:introduction}

この章では、本研究の目的と概要、本論文の構成について述べる。

\newpage

\section{研究の背景}\label{ux7814ux7a76ux306eux80ccux666f}

ヒューマンコンピュテーションやクラウドソーシングが流行していく中で、
計算資源としての人間の有用性に注目が集まっている。
計算機の能力だけでは処理が困難な問題を、人間を計算資源として利用することで
解決するヒューマンコンピュテーションは、今後より重要視されると考えられる。
ヒューマンコンピュテーションが一般的になるにつれ、計算資源としての
人間と計算機は、処理を記述するプログラムから見て等価になっていく。
同じ計算資源としてプログラムから利用可能となることで、人間と計算機の処理は
融合して記述されていくと考えられる。

しかし、既存のヒューマンコンピュテーションの仕組みでは、人間と計算機の処理の融合の実現は不十分である。
これらの仕組みの多くはクラウドソーシングプラットフォームを利用することが前提となっているが、
クラウドソーシングプラットフォームには「インターネットを介して、不特定多数の人間を使う」という特徴がある。
その結果、これらが応用される対象は人間の知能のみを利用した、
誰でもできるような単純な作業が中心となってしまう。

計算機と比較すると、人間は知能だけでなく、自由に動かせる身体を持つという優位性がある。
これらの要素を複合的に利用することではじめて、人間を最大限に活用できると言えるだろう。
人間にしかできないことの一部のみしか実現できない既存のクラウドソーシングベースの仕組みでは、
人間の能力を最大限に活用した人力処理を実現しているとは言えない。

\section{研究の目的}\label{ux7814ux7a76ux306eux76eeux7684}

本研究の目的は、人間と計算機の処理を融合させたプログラミング環境の実現である。
汎用的なプログラミング言語の上で、具体的な人間への指示と指示に対する処理結果の取得を可能にすることで、
計算機による処理と人間にしかできない処理を融合させたコンピュータプログラムが記述・実行できるようになる。
そうすることにより、今までのヒューマンコンピュテーションの枠組みでは
対象としていなかったような領域までをも処理できるようになり、有用なシステムの構築が可能となると考えられる。

本論文では、人間と計算機の処理を融合させたプログラミング環境の設計思想や具体的な実装について述べ、
どのような応用領域が存在するかを示す。
また、インタビュー調査の結果やシステムの有用な点、問題点についての考察を述べる。

\section{本論文の構成}\label{ux672cux8ad6ux6587ux306eux69cbux6210}

第\ref{chap:background}章では、計算資源としての人間の現状についてまとめ、
人間と計算機への指示を融合させた新しいプログラミング環境について説明する。

第\ref{chap:related}章では、関連する研究を分野ごとにまとめ、その特徴を述べる。

第\ref{chap:design}章では、提案する新しいプログラミング環境に求められる要件についてまとめる。

第\ref{chap:implementation}章では、第\ref{chap:design}章で検討したプログラミング環境の具体例として
Babascriptプログラミング環境を提案し、このプログラミング環境を構成する各要素の実装について述べる。

第\ref{chap:application}章では、提案するプログラミング環境によって実現が可能と考えられる応用例について考察する。

第\ref{chap:discussion}章では、インタビュー調査の結果について述べ、
本研究で提案するシステムにおける問題点や改善点等について考察する。

第\ref{chap:conclusion}章では、本研究の成果をまとめ、今後の展開について述べる。
