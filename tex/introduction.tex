\chapter{序論}\label{chap:introduction}

この章では、本研究の目的と概要、本論文の構成について述べる。

\newpage

\section{研究の背景}\label{ux7814ux7a76ux306eux80ccux666f}

ヒューマンコンピュテーションやクラウドソーシングが流行していく中で、
計算資源としての人間の有用性に注目が集まっている。
計算資源とは、コンピュータプログラムがその計算のために利用するあらゆる資源を示す。
CPUやメモリはもちろん、入出力装置等もその範疇に入る。
計算機の能力だけでは処理が困難な問題を人間を計算資源として利用することで
解決するヒューマンコンピュテーションは、今後より重要視されると考えられる。
ヒューマンコンピュテーションが一般的になるにつれ、計算資源としての
人間と計算機は、処理を記述するプログラムから見て等価になる。
同じ計算資源としてプログラムから利用可能となることで、人間と計算機の処理は
融合して記述されていくと考えられる。

しかし、既存のヒューマンコンピュテーションの仕組みでは、人間と計算機の処理の融合の実現は不十分である。
既存の仕組みでは、クラウドソーシングプラットフォームを利用することが前提となっている。
システムの多くがインターネットを介した不特定の人間が対象としているため、
人間の知能のみを利用した誰でもできるような単純な作業への応用が主である。
しかし、人間は知能のみではなくその身体や五感も優れている。
これらの要素を複合的に利用することで、人間の能力を最大限に活用できる。
人間の能力の一部だけしか利用できない既存のクラウドソーシングベースの仕組みでは、
人間の能力を最大限に活用した人力処理を実現しているとは言えない。

\section{研究の目的}\label{ux7814ux7a76ux306eux76eeux7684}

本研究の目的は、人間と計算機の処理を融合させたプログラミング環境の実現である。
従来のプログラミング言語上で具体的な人間への指示を可能にすることで、
より高度な人間と計算機の処理の融合が実現できる。
今までのヒューマンコンピュテーションの枠組みでは対象としていなかったような領域までをも処理できるようになり、
有用なシステムの構築を可能となる。

具体的には、プログラム上で人間とのインタラクションを可能にするプログラムモジュールであるBabascriptと、
プログラムと人間の仲介となるソフトウェアエージェントであるBabascript
Agentを組み合わせる。
分散処理プラットフォームであるNode-Lindaによって指示の配信を実現する。
これらの要素を総合して、Babascriptプログラミング環境と呼ぶ。

本論文では、Babascriptプログラミング環境の設計思想や具体的な実装について述べ、
どのような応用領域が存在するかを示し、有用な点や問題点について述べる。

\section{本論文の構成}\label{ux672cux8ad6ux6587ux306eux69cbux6210}

第\ref{chap:background}章では、プログラムの実行対象の広がりや、プログラムが処理する領域の変化に
焦点を当て、これらの変化によって実現が可能となった、人と計算機への指示を融合させたプログラミング環境について述べる。

第\ref{chap:related}章では、関連する研究を分野ごとにまとめ、説明を行う。

第\ref{chap:design}章では、提案する新しいプログラミング環境に求められる要件についてまとめる。
また、導き出した要件を元に、新しいプログラミング環境に求められるシステムについて考察する。

第\ref{chap:implementation}章では、第\ref{chap:design}章で検討した新しいプログラミング環境の具体例として
Babascriptプログラミング環境を提案し、このプログラミング環境を構成する各要素の実装について述べる。

第\ref{chap:application}章では、人間と計算機への指示を融合させた新しいプログラミング環境によって実現が可能と考えられる
新しい応用例について考察する。

第\ref{chap:discussion}章では、本研究で提案するシステムにおける問題点や改善点等について考察する。

第\ref{chap:conclusion}章では、本研究の成果をまとめ、今後の展開について述べる。
