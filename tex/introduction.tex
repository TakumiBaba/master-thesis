\chapter{序論}\label{chap:introduction}

\section{研究の目的と概要}\label{ux7814ux7a76ux306eux76eeux7684ux3068ux6982ux8981}

プログラムは実行したい処理を記述するためのフォーマットである。
コンピュータに実行させる用途として作られたため、コンピュータが理解できるよう、論理的に緻密に書かなくてはならない。
一方で、人間にも理解できる形で記述することが望まれている。
これはつまり、人間にもコンピュータにも理解出来るようなプログラムが優れたプログラムであるということだ。
プログラムはただコンピュータに対する命令を記述するものではない。
実現させたい状態に至るまでの過程を記述するものである。

我々はプログラムによるコンピュータの制御の恩恵を受け、様々なことに活かしている。
ちょっとした処理の自動化や、Web上でいつでも使える便利なサービス、スマートフォン上で動作する
アプリケーション、センサーやアクチュエータを使って自分の部屋の状態を変更させるなど
プログラムで記述できることは非常に有用である。

だが、プログラムで記述できる領域はまだまだ広く存在すると考えられる。
例えば、実世界で行うべきタスクをプログラムで記述しても、それを実行することは困難なことが多い。
その動作を実現できるセンサーやアクチュエータがなければ困難であるし、既存のコンピュータやプログラムだけでは
解釈することが難しいこともある。
そこで、センサーやアクチュエータ、コンピュータ等の既存の計算資源のみでは実現が困難な処理の実行対象として、人間に着目した。
人間は非常に優れた入出力及び演算装置としてみなすことができる。
この人間を積極的にプログラムに組み込んでいくことによって、人間というモジュールが存在しないが故に記述できなかった実世界で行うべきタスクでも
プログラムとして記述できるようになる。
本研究では、このようなプログラムを記述・実行できるような環境について提案する。

\section{用語定義}\label{ux7528ux8a9eux5b9aux7fa9}

本論文において使用する用語を以下のように定義する。

\paragraph{計算資源}\label{ux8a08ux7b97ux8cc7ux6e90}

計算資源とは、プログラムが実行中に利用可能な機器類を示す。
本研究においては、人間も計算資源として扱う。

\paragraph{ワーカー}\label{ux30efux30fcux30abux30fc}

プログラムからの指示内容を実行する人間を示す。

\paragraph{ソフトウェアエージェント}\label{ux30bdux30d5ux30c8ux30a6ux30a7ux30a2ux30a8ux30fcux30b8ux30a7ux30f3ux30c8}

ユーザとソフトウェアの

\section{本論文の構成}\label{ux672cux8ad6ux6587ux306eux69cbux6210}

第\ref{chap:background}章では、背景となるプログラムの意味について整理し、プログラムが記述する領域について述べる。

第\ref{chap:design}章では、本研究で提案する人と計算機の処理を融合させたプログラミング環境に求められる要件についてまとめる。
導き出した要件を元に、新しいプログラミング環境に求められるシステムについて考察する。

第\ref{chap:implementation}章では、第\ref{chap:design}章で検討した新しいプログラミング環境の具体例として
Babascriptプログラミング環境を提案し、このプログラミング環境を構成する各要素の実装について述べる。

第\ref{chap:evaluation}章では、システム評価について述べる。

第\ref{chap:application}章では、本提案で実現するプログラミング環境を利用して実現可能と考えられる応用例について述べる。

第\ref{chap:discussion}章では、本提案に関する考察について述べる。

第\ref{chap:related}章では、関連する研究分野についてまとめる。

第\ref{chap:conclusion}章では、本研究の成果をまとめ、今後の展開について述べる。
