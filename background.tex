\chapter{背景}
\label{chap:background}

\section{研究背景}

- HumanComputationの登場
  - 人間は明確に計算資源となる
- 人と計算機を同じように扱う
  - 人と計算機、双方に対する処理命令フォーマットが異なる
    - 計算機はプログラムの通りに動く
    - 人はマニュアル等に沿って動くことが多い
  - 実行可能なフォーマットに統一するべきでは
    - プログラムで記述できる
  - マニュアル等は非常にプログラム的
    - 例えば、運動会プログラムとか
  - 社会の多くはプログラムによって支配されている
- プログラムから人を扱う
  - 様々な研究
  - クラウドソーシング系のばっかり
  - 演算のための機能としてしか利用されることはない
- 人は汎用的実行主体である
  - 演算だけでなく、様々なことができるし、している
    - マニュアルは様々なことが記述されている
  - より需要があるのは、自分自身や家族、会社などの組織内の人間
    - 身近な人間をプログラムできるほうが良い
  - 普通にプログラムを書いてるかのように扱えるべき
- 特定の身近な人をプログラムに組み込めるような仕組みはない
  -
